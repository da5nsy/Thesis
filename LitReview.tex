\chapter{Literature Review}
\label{LitReview}

This chapter shall provide summaries of the background information required to understand the content of this thesis, and also review the state of the art of the various subjects that this thesis covers.

\section{Colour Science}

\subsection{Illumination and Colour Vision}

Organisms on earth posses visual systems such that they can glean information about spatially remote objects through the sensing of light reflected from these objects towards the organism. They are able to do this because the sun emits electromagnetic radiation (which we call `light' when it is within our visual sensitivity range), our atmosphere transmits many parts of this radiation (absorbing some wavelengths more-so than others), and objects absorb some and reflect some of this radiation (again, some wavelengths more-so than others).
%minnaert

\begin{figure}[htbp]
\includegraphics[max width=\textwidth]{figs/LitRev/daylightAndBananas.pdf}
\caption{The \gls{SPD} of a single measurement of daylight (sunlight plus scattered blue-sky light, from \citet{hernandez-andres_color_2001}) and the light reflected from 2 different surfaces - a green banana and a yellow banana (data from personal correspondence with David Slaughter after \citet{li_optical_1997}), computed by multiplying the \gls{SPD} by the measured \glspl{SRF} of the two surfaces. \Glspl{SPD} normalised such that the max of the daylight \gls{SPD} is 1.}
\label{fig:SPD}
\end{figure}

Our perception of colour generally correlates with the way in which objects preferentially reflect some wavelengths over others (described by the \acrfull{SRF}) which assists in the recognition of objects (that's a banana) and the discrimination of distinct objects, often in a manner that it ecologically beneficial (that's a \emph{ripe} banana).

\subsection{Chromatic Adaptation and Colour Constancy}
`Adaptation' is the general mechanism by which a finite range of sensitivity can be shifted in terms of absolute sensitivity bounds. The benefit of having an adaptive system, as opposed to a fixed system, is that the sensitivity of the system to small changes is maximised, whilst maintaining a broad overall sensitivity, at the expense of being able to sense over the entire range at a single time-point. 

\begin{figure}[htbp]
%\includegraphics[max width=\textwidth]{figs/litReview/???.png}
\caption{A comparison between a system with fixed sensitivity vs. one with an adaptive sensitivity.}
\label{fig:adaptation}
\end{figure}

In an environment such as the terrestrial environment, there is a great range in the level of illumination, but this range is rarely existent contiguously; levels of illumination tend to similar across a scene, and only change rather slowly. The notable exception, and thus where we notice the expense of having an adaptive visual system, comes when we enter or exit an environment where illumination is almost entirely excluded, such as a dark cave or below decks of a boat. 

Where the process of adaptation responds to overall illumination levels, it is referred to as light adaptation and dark adaptation. Where the process of adaptation responds to the chromaticity of the ambient illumination it is referred to as chromatic adaptation\footnote{There are a large number of simplifications and assumptions in this sentence. Firstly, adaptation will occur to the light which enters the eye, whether this is the result of an ambient illumination or not. Secondly, it is not clear that adaptation responds specifically to the chromaticity of illumination; it is likely more complex than this. The questioning of these assumptions will be a primary goal in this thesis.}.

%I am indebted to comprehensive overview papers on this subject by Foster37 and Smithson38.
Colour constancy refers to the stable perception of object colour appearance, in spite of a change in illumination which would cause a change in the nature of the stimuli reaching an observer\footnote{The terms `chromatic adaptation' and `colour constancy' are often used interchangeably; within this thesis I shall use `chromatic adaptation' to refer to an adaptive mechanism, and `colour constancy' as the ability possessed by an organism (which may well be bestowed \emph{by} chromatic adaptation).}. This objective change in stimuli seems to be effectively but not completely discounted by the human visual system; in order to maintain a stable perception of object appearance. 

In a natural environment, a change in the radiation reaching the eye from an object might be caused by changing weather or time of day, or by the physical movement of the object from one space to another (where different lighting exists in the two different spaces). It is the popular understanding that when the illumination changes, the colour of an object (to use the term `colour' in its common vernacular form, as representing an object attribute) doesn't change. This understanding holds true for all but the most extreme and/or artificial illuminations. Figure \ref{fig:fosterflowers} shows a single scene, under illumination of two different \glspl{CCT}, and the corresponding radiance spectra for a single point in the scene. The appearance of the object under the two illuminants is likely to have remained somewhat stable to an observer within the real scene, with observers choosing similar colour names for the object under each condition\footnote{Of course the appearance here, with these two images side-by-side, highlights the differences between the two conditions, and shows that under a single state of adaptation there is a clear distinction.}. If however, one asked an observer whether the \emph{appearance} of an object has changed between the two conditions, the observer would generally say that it has. We seem to have access to both the `raw input signals' and some estimate of the underlying \gls{SRF}.

\begin{figure}[htbp]
\includegraphics[max width=\textwidth]{figs/LitRev/fosterflowers.png}
\caption{A single scene under two illuminants (\Gls{CCT} = 4000k and 25000K), reproduced from \citet{foster_color_2011}.}
\label{fig:fosterflowers}
\end{figure}

Figure \ref{fig:greyballs} gives an indication of the extent to which an object (or more precisely in this case, a group of nominally identical objects) can have varying colorimetric attributes across relatively minor spatial variation. A similar level of variation occurs across time, for example as the sun passes behind a cloud.

\begin{figure}[htbp]
\includegraphics[max width=\textwidth]{figs/LitRev/greyballs.png}
\caption{Reproduced from \citet{nascimento_spatial_2014}. Original figure caption: ``The color rendition on the left shows the scene with six embedded probe spheres indicated by numbers. The dot plot [on the right] shows the distribution of the CCTs of the local illumination at the 17 sample points on each of the spheres.''}
\label{fig:greyballs}
\end{figure}

And so we have the central problem of colour constancy and chromatic adaptation: objects of which we have stable colour perceptions are able to be variant in both radiometric and colorimetric attributes without losing their apparent stability. The study of colour constancy aims to answer many questions which revolve around this central conundrum.

Traditionally %(pre-1986, when \citet{arend_simultaneous_1986} performed what \citet{foster_color_2011} refers to as `the first systematic behavioral experiments' on color constancy) 
there have existed two stances regards the nature of colour constancy; one held that it was enabled by adaptation of the sensory system, and the other that it was the result of unconscious inference. It is my understanding that whilst great progress has been made in the intervening three decades, this is still a central question, although it appears to be widely accepted that the two frameworks might be complementary rather than mutually exclusive. 

Johannes von Kries was one of the first to deeply consider colour constancy, and appears to have understood it as a problem of sensory adaptation. In his formative series of papers, \citet{von_kries_beitrag_1970} laid out how the study of chromatic adaptation could be divided into two broad and complementary aims. 

The first referred to the systematic representation of the transforms in sensation caused by specific adaptations, such that `for any light mixture stimulating the readapted part [of the retina], another light mixture is specified that stimulates the same sensation is a normally adapted part \dots The purpose of this study would be solved completely if a general rule could be obtained for the effects of all possible adaptions.' The search for `corresponding colours' (pairs of colours which match under asymmetric adaptation states), and for suitable systems to predict the appearance of colours in all situations, fall within this category of exploration. Out of this aim grew the study of \glspl{CAT}, where the search for an algorithm which would mathematically predict the locations in colour space of corresponding colours has been the fundamental goal \citep{cie_cie_2004-1}.

von Kries suggested that the second categorical aim of chromatic adaptation research ought to be to discover `how the adaptation is produced by exposure to any particular color, continued over an extended period of time.' 

von Kries' distinction carves a divide between the study of the mechanisms of chromatic adaptation, and the effects of chromatic adaptation. It would be unreasonable to think of these areas of study as unrelated, but it seems reasonable to recognise their separability. The best justification for their separation is perhaps that success in their study might serve different aims. The first, the prediction of colour appearance, serves to further the utility of colour science, in particular colour appearance modelling, and the other systems which rely upon colour appearance modelling, such as colour difference formulation. 
The second is best considered a vision science problem, concerned as it is with the function of the human visual organ, and advancement might endow us with an extension in our understanding of the human body. This in turn would have benefit in multiple areas, including feeding back to the prior aim, for once we understand the underlying mechanics we stand a much better chance of creating robust predictors of colour appearance. 

%Of particular interest to this author, understanding the underlying mechanisms of colour constancy should allow for a more intelligent design of lighting for the spaces which we inhabit, considering that artificial lighting need only resemble natural lighting to the extent that the visual system treats it as similar enough. Creating a colorimetric match to a specific spectra such as D65 is relatively easy, but until chromatic adaptation is more thoroughly understood it is not necessarily entirely clear what the appearance of such a match, or objects illuminated by it, might be. 

In the same set series of papers, von Kries laid out his understanding of chromatic adaptation, in terms of both how corresponding colours might be calculated, and how one might consider the underlying mechanisms which allow for colour constancy might operate. Here I shall abridge some of his writings to present the most salient insights, and attempt to explain what is meant by a `von Kries transform' in modern parlance.

von Kries noted that `light mixtures that appear matched to the white-adapted eye always remain matched to the eye when it is adapted in any other manner.' This is known to be only partly true, but when given the restriction of referring to photopic vision solely, this assertion becomes more palatable. If this statement is respected, the implications are rather grand; principally the result of this verdict would be that the spectral sensitivities of the cone cells do not change as a result of chromatic adaptation. If they were to change in some way, then distinct mixtures of light which match in one adaptive state might fail to match under a different state. von Kries referred to this idea as the `persistence rule.'

The second rule which von Kries presents is referred to as the `theorem of proportionality'. This theorem describes how if two pairs of corresponding colours are additively combined for their respective adaptive states, the resulting colours will also be corresponding colours. 
It follows from the von Kries' theorem of proportionality that increasing or decreasing the luminance of corresponding stimuli shouldn't void their equality, though of course this statement relies even more heavily on the caveat that this should only be assumed for purely photopic vision.

von Kries goes on to suggest that it also follows, if considered in combination with Grassmann's law \citep{grassmann_zur_1853}, that the conversion of any arbitrary light by exposure to conditions causing a specific chromatic adaptation, can be known if the nature of three other conversions are known. This relies on it being possible to consider any light as a linear combination of three others, and for the mechanism of chromatic adaptation to depend solely on the fatiguing of three independent systems.

von Kries' assertions may be mathematically described as the following:

\begin{subequations}
\begin{align}
L_{a}&=K_{L}L \\
M_{a}&=K_{M}M \\
S_{a}&=K_{S}S
\end{align}
\end{subequations}

where $L$, $M$ and $S$ represent the cone group responses; $K_{L}$, $K_{M}$ and $K_{S}$ represent distinct scalars and $L_{a}$, $M_{a}$ and $S_{a}$ represent the post adaptation cone group responses. In terms of what might set these scalars, or gain values, von Kries suggested that: 

\begin{quote}
`the organ of vision becomes less effective for that kind or for that part of its performance which is demanded from it for an extended period of time, whereas it becomes more effective for the activity which is, in a sense, opposed to that. This can be conceived in the sense that the individual components present in the organ of vision are completely independent of one another and each is fatigued or adapted exclusively according to its own function.'\citep{von_kries_beitrag_1970}
\end{quote}

% Quote from Fairchild about the endurance of von Kries?

von Kries' thoughts have inspired many decades of research into \glspl{CAT}, which are a vital part of modern colour appearance models and other colour science computations such as colour rendering index calculations. The investigation of \glspl{CAT} generally occurs through the fitting of algorithms to data of corresponding colours, collected in varying manners, such that corresponding colours might be predicted for specific situations. The principle divides between the types of \gls{CAT} tend to hinge upon the method used to collect the data43,44, which will be discussed further in the `Experimental Methods' section. Nayatani43 divides the set as those derived from chromatic adaptation theory and those derived from fitting to experimental data, whereas Luo44 makes the divide between those studies based on aperture based experimental procedures and non-aperture based experimental procedures (which generally incorporate more complex, and therefore arguably more realistic, stimuli). Both parties use the terms `CAT Type I/II' to refer to these distinctions, but it is not entirely clear whether their distinctions are complementary. 

This duality %get rid of this bollocks
is mirrored by Foster37 in his discussion of the experimental methods employed in colour constancy research where he describes fundamental differences between those experiments which probe chromatic adaptation by asking observers to make `paper matches' (make it look like these two patches were cut from the same sheet of paper) and h/s/b matches (where observers are asked to match appearance in a non-relational manner, that is to match the qualia of a stimuli). This duality occurs again in Foster's37 comparison of relational and non-relational colour constancy, where relational refers to the maintaining of colour relationships in a scene and non-relational again refers to abstracted stimuli qualia). It is my suspicion that all of these disparate dualities are representations of the fundamental duality considered at the beginning of this chapter; that of a distinction between sensory adaptation and inferential computation.

%big jump
Considered in the context on von Kries' work, Edwin Land's Retinex model45,46,47 might be described as `von Kries with spatial considerations' but I imagine such a statement would have rather riled the often bombastic sounding Land. It appears to have been Land's understanding that the Retinex model was not a model of chromatic adaptation, but rather a model to usurp and do away with the very concept of chromatic adaptation.

Land was fond of picking apart Newton's statements regarding colour; particularly that the perception of colour was the result of an objects' `excess and predominance in the [spectra of the] reflected light'48. Land asserted that rather than work from the premise that the visual system was correcting an absolute record of the world, by normalising it to account for the ambient light source, one should instead consider visual input only as a relative record, where each element within a scene only takes on properties by comparison with the other elements of the scene. 

He suggested for consideration the idea of the human visual system recording three separate lightness images, each representing the recording from a different band of receptors, where each element within each image is scaled against the brightest element in each specific image. Retinex theory also provides an interesting algorithm for discounting not only coloured illumination but illumination which varies across the visual field.

% other theories?
% throw in the imaging folks here

\subsubsection{Experimental Methods for Colour Constancy Research}

A large range of experimental methods have been used to investigate the problem of colour constancy, partly because different experimenters were approaching the problem from different angles aiming to accomplish subtly separate goals.
Those approaching from a mathematical angle (perhaps with the mind-set `if we find the algorithm which predicts corresponding colours a- this is very useful for industry and b- we can then work backwards to explain how human colour constancy may occur') are concerned primarily with producing chromatic adaptation transforms.
Those approaching colour constancy with a basic interest in the physiology and general function of colour constancy have adopted/created a rather wider set of experimental techniques, owing to the fact that this direction of study is much less well defined than the former, and is at a more exploratory stage of its development.
The CIE TC 1-52 report (CIE 160:200442) includes an interesting note on the distinction between these two broad approaches; an appendix discusses the reasons why a concerted recommendation was not reached by the group lists the principal factor being a distinction between perspectives as described above. 
CIE TC 1-5242 and Luo44 provide a comprehensive overview of the different methods which have been used by the CAT group:
1.	Haploscopic matching
2.	Local Adaptation Matching
3.	Memory Matching
4.	Magnitude Estimation
Haploscopic Matching is the most common technique in this field, and the term refers to experiments which differentially adapt the two eyes and allow an observer to vary attributes of the stimulus presented to one eye such that it in some way matches the attributes of a fixed stimulus shown to another eye. Whilst this is in many ways unnatural, the benefits are that an experiment can be set up so that there is no time interference (the presentations are simultaneous, so memory effects are avoided) and that high precision of match is relatively easily achieved. As assumption is made that the adaptation of each eye is independent.
 
Figure 11 From42.
Local Adaptation Matching can be considered as a variation of haploscopic matching where instead of differing adaptational stimuli being presented to each eye, differing adaptational stimuli are presented to different parts of the same eye (spatially distinct areas of the same retina). The assumption here is that there is minimal intra-retinal interaction. MacAdam's 1956 study49 epitomises this technique. This experimental technique requires that observers minimise eye movement, in order to maintain spatially distinct adaptation.
Memory Matching has traditionally been performed by training observers to communicate colour sensation through the munsell system notation50,51, and then asking observers adapted to different ambient lighting to describe set real objects using such notation. This technique is not much used due to many limitations and confounding factors. Luo44 details the limitations  of this experimental technique succinctly: `a substantial training period being required, complicated procedures for data analysis, lower precision than that of haploscopic technique, limited capacity for retaining information, and memory distortion.'
Magnitude Estimation appears to be similar to memory matching, in that observers are requested to verbally describe an object whilst in an ambient adapting field. The distinction is that the observers are requested to communicate their perceptions using the perceptually meaningful attributes of hue, saturation and brightness, and as such results can be easily integrated into colour appearance models. Recent experiments52,53,54,55,56,57 collected data which was used to create CIECAM97s.
Away from the development of CATs, a subtly different set of experimental techniques has been developed with which to probe the operation of colour constancy. An admirable overview is provided by Foster37:
1.	Asymmetric Matching
2.	Colour Naming
3.	Achromatic Adjustment
4.	Discriminating Illuminant Changes from Reflectance Changes
Asymmetric matching is in many ways synonymous to the haploscopic matching described above and by in CIE 160:200442, in that it describes an experimental set-up whereby one stimuli is compared with another, generally where each stimuli exists within a distinct adapting field and attributes of one stimuli can be either adjusted or responded to by an observer. The term asymmetric matching might be thought to be inclusive of a wider range of experimental set-ups, where haploscopic (Greek roots: haploieides, single and skopeo, to view) is necessarily concerned with each individual eye receiving distinct input. Asymmetric matching may refer to experiments where stimuli are viewed simultaneously, successively, or in an alternating fashion, binocularly or haploscopically.
Colour Naming is a technique employed with the aim being a more natural task than  asymmetric matching, and removing some of the `instruction effects' probed by Arend and Reeves40. Foster argues37 that colour naming represents a task apt to measure colour constancy more directly, as opposed to the `relational colour constancy' often studied in asymmetric matching experiments, since it concerns identification rather than equivalence. One clear benefit seems to be that the observer needn't be aware of the equivalence; it is expected that in such experiments there will be only one stimulus, perhaps with a temporally variant adapting field. An observer is simply asked to name colours, and this should theoretically result in a measure of adaptational colour constancy as opposed to inferential colour constancy, so long as the stimuli is suitably abstract . Colour names may be of a fixed set, or an observer may be given free choice. Analysis of results can employ a naming centroid based approach or a boundary focused approach. Speigle and Brainard58 proposed a novel approach combining magnitude estimation and colour naming with the aim to improve precision of response.
Achromatic Adjustment experiments ask an observer to in some way set a stimulus to a neutral achromatic tone, on the assumption that the internal grey point of an observer shifts in response to different adapting fields. These adjustments are generally easy for an observer to make, but the extrapolation of the results makes various assumptions about conceptual colour space and the nature of achromacy. Experiments are easily confounded by complex or real stimuli where there exists a close-to-neutral object in the scene which could consciously or unconsciously be used as a reference.
Discriminating illuminant changes from reflectance changes provides a key way to examine colour constancy in an operational manner. Following the assumption that chromatic adaptation allows an observer to discount the illuminant in some manner, an experimental set up where observers are requested to distinguish between an illuminant change and a reflectance change represents a situation which very closely mirrors the natural process of colour constancy. This experimental technique is well placed to examine whether or not colour constancy in this form is active and efficient, but it provides little way of probing the underlying mechanisms of colour constancy.

%recent advances in colour constancy?
% using more realistic stimuli
% the grey edges algo
% the comparison of grey edge and grey world/bright is white


\subsection{Colorimetry and Colour Measurement}

\textit{The current recommended source for colorimetry is CIE 015:2018 \citep{cie_cie_2018}\footnote{Though as \citet{fairchild_cie_2019} notes, this document is `expensive and somewhat difficult to find', and as such I have been using a draft of the now superseded \citet{cie_cie_2004-2}.}. My personal opinion is that an understanding of colorimetry of CIE colorimetry is best gained from an understanding of it's history, and for this I recommend Janos Schanda's book `Colorimetry: Understanding the CIE System'\citep{schanda_colorimetry_2007}.}

Colorimetry is the study of the quantitative specification of colour. As a subjective, internal and anthropocentric concept, in order to measure anything meaningful and comparable, we use a standard observer, or more precisely, one of a number of defined standard observers \cite{cie_bs_2011}.

The classic standard observer was defined by the CIE in 1931, following experiments by Wright and Guild \cite{wright_re-determination_1929, guild_colorimetric_1931}. Despite several more recently published standard observers, the 1931 observer is still much used, and I shall use it in the following example of how a basic colorimetric computation is performed.

An illuminant is defined by its \gls{SPD}, a surface as it's \gls{SRF}, and a the sensitivity of a sensor (such as a photosensitive cell in the retina, or a pixel in a camera) by its \gls{CMF} (or in the case that biologically based measurements are used - the \gls{SSF}).

\begin{figure}[htbp]
\includegraphics[max width=\textwidth]{figs/LitRev/SPDetc.pdf}
\caption{An \gls{SPD} (D65), a set of \glspl{SRF} (from a Macbeth Colour Checker) and three \glspl{CMF} (CIE 1931).}
\label{fig:specFun}
\end{figure}
% How is D65 normalised here?

The light reaching the eye for a given reflecting surface under a given illuminant (termed the `colour signal') can be computed by multiplying the \gls{SPD} by the \gls{SRF} at each sampled interval.

\begin{equation}
\phi(\lambda)=R(\lambda) \cdot S(\lambda)
\end{equation}

Explain\dots \\
From this tristimulus values can then be computed.

\begin{subequations}
\begin{align}
X=k \sum_{\lambda} \phi_{\lambda}(\lambda) \overline{x}(\lambda) \Delta \lambda \\ 
Y=k \sum_{\lambda} \phi_{\lambda}(\lambda) \overline{y}(\lambda) \Delta \lambda \\ 
Z=k \sum_{\lambda} \phi_{\lambda}(\lambda) \overline{z}(\lambda) \Delta \lambda
\end{align}
\end{subequations}

Explain\dots \\
Following this chromaticity co-ordinates can be computed.

\begin{subequations}
\begin{align}
x=\frac{X}{X+Y+Z} \\
y=\frac{Y}{X+Y+Z} 
\end{align}
\end{subequations}

Explain\dots \\
Here's a \gls{MATLAB} example, using \gls{PTB} for data:

\begin{lstlisting}[language=MATLAB]
load spd_D65 % SPD: CIE D-series illuminant D65
load sur_macbeth % SRF: macbeth colour checker
load T_xyz1931 % CMF: CIE 1931

colourSignals = sur_macbeth.*spd_D65;
XYZ = T_xyz1931*colourSignals;
xy = [XYZ(1,:)./sum(XYZ);XYZ(2,:)./sum(XYZ)];
\end{lstlisting}

Plotting: 

\begin{lstlisting}[language=MATLAB]
spectralLocus = [T_xyz1931(1,:)./sum(T_xyz1931);T_xyz1931(2,:)./sum(T_xyz1931)];
sRGBSpectralLocus = XYZToSRGBPrimary(T_xyz1931);

figure, hold on, 
scatter(spectralLocus(1,1:70),spectralLocus(2,1:70),[],sRGBSpectralLocus(:,1:70)','filled')
scatter(xy(1,:),xy(2,:),'k')
axis equal, axis([0 1 0 1])
xticks([0 1]), yticks([0 1])
xlabel('x'), ylabel('y')
\end{lstlisting}

\begin{figure}[htbp]
\includegraphics[max width=\textwidth]{figs/LitRev/ColorimetryDemo1.pdf}
\caption{The CIE 1931 chromaticity diagram, showing the chromaticity of the macbeth colour checker patches under D65.}
\label{fig:1931}
\end{figure}

\subsubsection{Other Colour Spaces}

u'v' 

\gls{MB}

CIELAB and CIELUV

\subsection{Colour rendering and light quality specification}

Colour rendering indices provide an indication of the colour appearance of objects under a test illumination. Colour fidelity indices (properly a subset of colour rendering indices, however the most commonly used colour rendering index is in fact a colour fidelity index) are designed to describe how well a light source will produce a faithful appearance in terms of corresponding colorimetry to an appearance under CIE D-series illuminant (daylight proxy) or a black body radiator. Target values are generally provided in museum lighting guidelines \citep{ies_ies_1996,druzik_guidelines_2012,cibse_lighting_2015,thomson_museum_1986}. The recommended value for the ubiquitously used CIE R\textsubscript{a} (As defined in CIE 13.3:1995 \citep{cie_cie_1995}), often informally referred to as `CRI', varies from publication to publication, but is generally CIE R\textsubscript{a}$>$80, but with no experimental work referenced to support this figure. Research in colour rendering is at a point of change and development, with the recognition that a single figure might never be enough to describe the complex, multidimensional, subjective and context dependent qualia of colour rendering\citep{rea_color_2008}. Thus new indices, and modifications and addenda to long-established indices are being proposed \citep{smet_memory_2012,davis_color_2010,rea_practical_2010,color_metric_task_group_of_the_ies_ies_2015,teunissen_characterising_2016}.

To discuss the colour rendering of a light source is to discuss the colour appearance of objects which the light source's illumination may facilitate. 

If we were to take a bunch of balloons (or any objects, but balloons provide a simple and vivid image in the minds of most) of varied and multiple colours, from an area lit with daylight to an area lit with artificial light, and the colour appearance changes, for example the colours become more dull, we would generally be disappointed, understandably so. In this case, we could say that we were dissatisfied by the colour rendering qualities of the artificial light source. If the colours remained the same we probably would think little of it, but if we were pressed for a comment, we might say that the colour appearance was satisfactory.

If the balloons stayed the same colour what we'd be seeing specifically is a high level of colour fidelity, fidelity meaning faithfulness or truthfulness. If fidelity aligns well with the priorities of the user, then high levels of fidelity are desirable. It is assumed when discussing colour fidelity generally, that we're actually discussing fidelity to appearance under daylight or sometimes a Planckian radiator, aka a black body radiator. This is a fine comparison, but one which should be consciously considered; fidelity is not a measure of a light source per se, it is a comparison of that light source to a reference illuminant.

Fidelity has long been considered semi-analogous for colour rendering as a whole, but it should be remembered that it is really only one element of colour rendering, albeit one with useful correlation to others. Here I should point out that the CIE definition of colour rendering, from the publication CIE S 017/E:2011 ILV: International Lighting Vocabulary \citep{cie_cie_2011} reads: ``Effect of an illuminant on the colour appearance of objects by conscious or subconscious comparison with their colour appearance under a reference illuminant.'' You'll notice the similarities to the definition I have ascribed to colour fidelity, and specifically not colour rendering. This is an unfortunate clash of nomenclature which I see as inappropriate usage on the part of CIE, and at some point I hope this official definition will be updated to match the modern vernacular.

When considering how `good' a light source is at facilitating colour with the objects which it illuminates, we might consider different measures alongside or in place of fidelity.

\begin{itemize}
\item \emph{Colour Fidelity (to recap)} - A measure of comparison between how colours appear under daylight (or other reference) and how they appear under another light source. %Do the balloons appear the same colour indoors and out? Are individual colours important or is an average suitably representative?
\item \emph{Chromatic Discrimination} - The ability of a light source to illuminate objects such that subtle differences in colour are made most visible. %Can you tell the difference between all the balloons? How big are the differences? Are threshold differences more or less important than the perceived magnitude of suprathreshold differences?
\item \emph{Observer Preference} - Subjective preferences for the colour of objects, potentially related to an increase in saturation, or an observer's memory of colours, or specific colour effects (e.g. enhanced skin tones). %Do the balloons look `nicer'/more vivid/more like the last time you saw them?
\end{itemize}

These areas overlap, and often exhibit correlation. For example, if a light source has properties which allow for high fidelity rendering, quite often that light source might also allow for good colour discrimination and also facilitate colours which might appear natural and/or pleasing. These are not reliable correlations however; it is quite possible to have a light source with very dissimilar properties to daylight that could score highly on discrimination and/or preference, or vice versa.

The index in widest use currently for quantifying colour rendering is CIE R\textsubscript{a} (The General Colour Rendering Index). At start of my studies, this metric almost exclusively to generate the figure listed as `CRI' or similar on the packaging for bulbs available commercially. %In the intervening time, IES RP-30-15\cite{color_metric_task_group_of_the_ies_ies_2015}, and more recently IES RP-30-18\cite{ies_ies_2018}, have been published. %and...

CIE R\textsubscript{a} is a colour fidelity metric, produced by computing the colour differences of 8 specified test colour samples (TCSs), between the situation where illumination is provided by the test source and that where it is provided by a reference illuminant with the same \gls{CCT} as the test source, and taking the arithmetic mean of these colour differences. The index is scaled such that the highest score is 100, for light sources which reproduce the TCSs under the reference illuminant exactly, with any light source which induces a colour shift receiving a value less than 100, descending below 0 if the magnitude of the shifts are great enough.

The most recent version of the standard specifying this metric is `CIE 13.3-1995 Method of Measuring and Specifying Colour Rendering Properties of Light Sources' \citep{cie_cie_1995}. This document briefly overviews the history of, and necessity for, a colour rendering index and highlights areas of the procedure which vary compared to previous recommendations, and clearly details the recommended procedure. The process is summarised in the flow chart of Figure \ref{fig:criflow}. For a worked example see \citet[p.388]{hunt_measuring_2011}.

In response to a lack of confidence that CIE R\textsubscript{a} delivered meaningful values for LED illumination, a considerable amount of research and discussion has revolved around the subject of a colour rendering index to supersede CIE R\textsubscript{a}. This is commonly complicated by the assumption that the values obtained from a fidelity index in some way correlate with a visual appearance; CIE R\textsubscript{a} does not predict preference between two light sources. A more valid criticism of CIE R\textsubscript{a} is that it comprises deprecated methods and data sets for computing intermediary stages within the process of calculation, and thus this theoretically limits the `correctness' of any values produced by it.

Following many years where multiple CIE technical committees aimed to deliver updated colour rendering indices, but subsequently didn't make any firm recommendations, an IES group was formed with the aim of examining the issue. The output of this IES group was the production of IES TM-30-15\cite{color_metric_task_group_of_the_ies_ies_2015}, which follows the same philosophy as CIE R\textsubscript{a} but with more modern colour science. It is still a contentious issue whether this updated index genuinely offers advantages over CIE R\textsubscript{a}. 
% it also includes a bunch of other things...
% and IES TM-30-18?

Alternative propositions which offer fundamentally different approaches to the evaluation of colour rendering have been offered in recent years, including: 
\begin{itemize}
\item indices based on production of memory colours\citep{smet_memory_2012}.
\item indices which consider the size of the gamut of a TCS set\citep{rea_color_2008,teunissen_characterising_2016}.
\item indices which are fundamentally fidelity indices but which penalise or reward specific traits known to be disliked/preferred\citep{ohno_rationale_2010}.
\end{itemize}

\begin{figure}[htbp]
\includegraphics[max width=\textwidth]{figs/LitRev/criflow.png}
\caption{The stages of calculation for the computation of the CIE General Colour Rendering Index. Notes: Reference illuminant: the same or nearly the same \gls{CCT}. Below 5000K reference spectrum will be that of a black body radiator, and from 5000K `one of a series of spectral power distributions of phases of daylight'.}
\label{fig:criflow}
\end{figure}


\section{Museum Lighting}

\begin{quote}
`Museums and art galleries collect, preserve, and display natural artifacts and/or examples of human achievement and analyze their impact on the world and the universe around us. Effective exhibit lighting must balance exhibition and conservation needs and enrich the museum experience.'
\end{quote}

\begin{flushright}IES RP-30-96 Museum and Art Gallery Lighting: \\A Recommended Practice \citep{ies_ies_1996}\end{flushright}

Lighting in museums is required to satisfy multiple criteria; perhaps the least contestable requirement being that the lighting illuminate objects such that they are suitably visible to museum visitors. Also of upmost importance in most museum settings is that the lighting does not have an unreasonably deleterious effect upon the objects or environment, be this through direct photodegradation or as a result of heat transfer. Further to these requirements, an increased or optimal visual quality is generally desirable, although what this represents or how to achieve it is generally ambiguous.
Museum guidance documents provide advice on how to manage lighting to best address the above requirements and many other additional specific requirements through the recommendation of procedure and provision of target figures for quantitative variables. An overview of Museum Lighting Guidance Documents is presented in a later section of this document (3.1 Overview of Museum Lighting Specification Guidance Documents).
Theoretically, there exists a division between recommendations which deal with visual appearance and those which deal with physical degradation; the first group supported by the science of human vision, and the second supported by material sciences and chemistry. Whilst it is important to be aware of this distinction, it is normally not possible to consider them entirely separately in practice, as many variables which affect one actually affect both.
In sweeping terms, all electromagnetic radiation (visible and non-visible) damages objects, and more radiation damages objects proportionally moreso. Thus the question becomes: how little light can we use to illuminate objects such that they're visible to the extent required? The opposing question, sometimes used, on the assumption that more light always represents an increase in observer satisfaction/pleasure is: `how much light can we use so that only x damage occurs over y years'. The two are often used in combination or amalgam. It should be obvious that non-visible radiation, being of no visual benefit but having potential to harm, should be reduced either through using technology which doesn't produce it, or attenuating it wherever possible.
Extending the question of visibility is the subject of appearance. It is a general expectation that museums represent objects in a truthful and impartial manner, and it seems sensible that decisions concerning the appearance of items in museums should be made with this in mind. Alongside this, many museums treasure items where part of their value is aesthetic2, and it follows therefore that a technique which aids in the beautification of an object might be of interest, and this might be particularly of interest if the object is known to have deteriorated since the time of production.

\subsection{Lighting at The British Museum}

Lighting at the British Museum has developed in a somewhat organic manner, from the early days of the museum before the introduction of artificial lighting. This leaves the museum with more than sufficient daylight in most spaces during daylight hours, where the increased modern knowledge regarding the deleterious effect of lighting now means that conservators must find ways in which to limit this natural resource so that objects are not unduly exposed. 
The gallery lighting is replaced when a gallery is refurbished, and the latest technology is installed when a new gallery is created, where viable in terms of suitability and considering financial limitations. Other than this, lighting technologies tend to not be updated other than to replace individual lamps. This coupled with the grand scope of the museum, which means that it is rare for multiple galleries to be refurbished simultaneously, has led to a vast array of lighting designs and technologies. It in fact appears to represent a rather special example of `museum lighting through the ages', with daylight, tungsten, fluorescent and metal halide lamps all seemingly represented, as well as various illumination geometries reminiscent of the times of their fitting. It should be noted that these assertions are made following empirical observation with spectrophotometers and not from conversations with museum staff, see section 3.3 Collection of \gls{SPD} Data at British Museum.
There also seems to be a great range of lighting quality within the British Museum, with some spaces feeling bright and others comparatively gloomy. There is a range of colour temperatures and chromaticities of light at the museum, as shown in Figure 1.

Multiple methods are currently employed to limit the exposure of objects in the museum. UV absorbing film or glazing which incorporates UV reduction is used throughout the museum and in certain galleries there are automatic blinds which limit the intrusion of direct sunlight at specific times of the day and year. For particularly sensitive objects, rooms are lit artificially at very low levels, and some objects are selectively lit in order to further limit their accumulative exposure.

\subsection{Visitor Requirements of Museum Lighting}

The visitor requirements of museum lighting are understandably varied and complex. The examination and definition of these requirements has been the subject of a small number of studies, principally by Kesner3 in the USA in the early nineteen-nineties. These studies were based on survey based self-reporting, and concluded that colour accuracy was the highest requirement and that `richness' of colour was the lowest priority. The meaningfulness of these conclusions is cast into doubt by the high average scores for each category (>4/5 for each category).
\begin{quote}
Artifact appearance, particularly clarity of artefact form and accuracy of artifact color, is the most important visitor need. Although visual impression, specifically acceptable gallery brightness and rich artefact color, is least important among the factors, it too rated highly important.3
\end{quote}
Further studies, spurred by recent adoption of LED technology have been completed, although with very limited and questionably scientific basis at the Field Museum4.
Additionally work has been completed5 which describes the results of a written survey of museum professionals on the subject of lighting specification, focussing on SSL (effectively LED) adoption, and their justification for specifying specific lighting. 
979 people who had requested access to the guidelines document produced by some of the same authors6 were emailed with a request for participation. 46 (4.7\%) responded with a completed survey. Of those, 30\% are described as `international' which considering the base of the authors is in the USA we can assume as meaning `not based in the USA'. The survey was prepared as part of the USA government's GATEWAY program, and can be considered as a follow-up to work on recommendations for specification6. 
The headline results are as follows:
-	When the `success' of new LED based lighting schemes was evaluated the public reaction was either neutral or positive in all cases. 
o	There appears to be an issue with the data for this question however, with the `no response' garnering 48\% of the responses and `favorable response' garnering 55\%.
-	The distribution of staff response was broadly similar: `no response'-17\%, `negative response'-3\%, `favorable response'-80\%.
-	The results of two priority ranking questions  are broadly similar:
o	First question: The first priority is identified as `limiting damage' with other priorities broadly equal.
o	Second question: Joint first priorities are identified as `colour and spectral power distribution' and `damage potential' with others broadly similar (though dimming, input power and warranty lagging behind others)
-	91\% of respondents reported using `Color Rendering Index (CRI)' which is assumed to refer to CIE Ra. 6\% state that they refer to other Ri values, and 20\% report using CQS.
-	60\% of respondents report that they consider \gls{CCT}.



\subsection{Current practise in specifying museum lighting}

\subsubsection{Overview of Museum Lighting Specification Guidance Documents}

An overview was written of five of the most prominent and referred to museum lighting guidance documents:
•	The Museum Environment8
•	CIE 157:2004 Control of Damage to Museum Objects by Optical Radiation59
•	Guidelines for Selecting Solid-State Lighting for Museums6
•	SLL LG8: Lighting for museums and art galleries7
•	IES RP-30-96 Museum and Art Gallery Lighting: A Recommended Practice1
The main conclusions were:
-	Most recommend following values for CRI/\gls{CCT} but also suggested visual inspection 
-	Recommended values were provided for:
o	Lux (various, dependent on sensitivity, generally based on figures from the study of Loe et al.68)
o	CRI (various, most frequently <80, with no experimental basis referenced)
o	\gls{CCT} (various, based implicitly on Kruithof67 or on such ideas found empirically)
-	There was often blurring between recommendations concerned with visual appearance and those concerned with conservation. 

\subsubsection{Interviews with Museum Professionals}

In order to gain an understanding of how museum professionals specify lighting, interviews were conducted with 12 museum professionals representing 10 UK based museums, galleries or historic property management groups, in spring of 2016. A small number of these professionals represented UK-wide groups, but the majority represented London based institutions.
The interviews were semi-structured around a set of questions composed by my supervisors and I, which are included as Appendix 1: Interview Questions in the present document. The choice to conduct these interviews in a semi-structured format as opposed to a fully-structured one was based on the desire to allow unanticipated topics to enter the conversation, so as to limit the potential for important subjects to be neglected due to naivety of foresight. The conversational format of the interviews meant that the resulting data was qualitative rather than quantitative, which to an extent hinders comparison, but it was decided that the variety of job roles and institutional sizes, and the small sample size, already made quantitative comparison of limited meaningfulness for this investigation.
The following discussion shall consider the results in roughly the same order as the questions were presented to the interviewees, which is the same as the order of questions in Appendix 1: Interview Questions. Although specific questions shan't be referred to, it may be useful to read this section alongside the appendix.
The interviewees were contacted through introductions from supervisors, personal connections, or cold emails. Almost all held conservation based roles (the exception being an `exhibitions manager') and noted that their principal responsibility regards lighting was controlling the safety of lighting regards its ability to cause damage to the objects held and displayed by their respective institutions. This generally involved monitoring and analysis of existing lighting systems and natural illumination, creating general guidance documents for the specification of lighting in their specific institutions and for loan items, and providing guidance and recommendations for the fitting out of new galleries or gallery refits. Few considered themselves directly responsible for the appearance of museum objects, considering this to be a creative decision outside of their remits. Future work considering the perspective of others involved in museum lighting, such as representatives from external lighting design companies, is required.
Whilst a small number considered themselves active in the area of lighting research, all interviewees were responsible for more conservation considerations aside from lighting, limiting the practicable level of specialisation. Many considered communication and dissemination a key part of their role, often noting they often found themselves in the position of needing to educate other teams within their institution on subjects including lighting. 
`This is not my science, my job is pulling it out and presenting it to others'.
Asked what each considered to be `good lighting', responses included `safe', `invisible' (you don't notice it), and `lighting which is appropriate for your objects and your exhibition' (noting the variability of requirements dependent on the particular object(s) being presented). Asked to present a list or range of priorities, many focused on the safety requirements of lighting. The principal safety concern for lighting was that it fell below specific illuminance criteria, dependent on the assumed sensitivity category of the object in question. The specific target values were generally those provided by Thomson8 of 50lx for sensitive items and 200lx for less sensitive items, which are based on the visual preference work of Loe68.
Considering a scale of other priorities, following the requirement for appropriate lux levels, considerations included: limiting/excluding UV, obtaining an acceptable CRI value, and time and capital costs associating with fitting and maintenance. Energy efficiency was also a driver, but this did not seem to be a factor within technology category, rather it was noted that many institutions are switching to LED because of increased efficiency, but that the difference in efficiency between one LED type and another was negligible compared to the saving in contrast to lighting technology which LED was replacing, which was often tungsten.
The range of roles played in the procurement of lighting varied amongst those interviewed. Whilst some created guidance documents which would then be passed on to estates teams and exhibition designers or lighting designers specifically, others had a much more hands-on role, testing specific lighting before installation or making recommendations on a case-by-case basis. It was noted that whilst relamping and retrofitting was generally handled by `in house' teams, where new galleries or large temporary exhibitions were created it was common for an external lighting design company to be contracted to perform the work. When asked whether recommendations were normally followed, the general reply was that recommendations for lux exposure and UV content were almost always followed, but other recommendations (such as for CRI or \gls{CCT}) were more loosely interpreted. In many cases recommendations for \gls{CCT} were not made.
When asked about the tools used to make such recommendations and choices, responses included references to guidelines and reference sources (most often 8, though also 6 and 69), references to specific units such as `lux' used in tandem with recommendations from guidelines such as above, indices such as CIE Ra (referred ubiquitously to as `CRI') and notes of specific conferences which were attended in order to stay up to date with the research on the subject. There was a general feeling that the current climate was one of swift technological change in lighting, which created an increased difficulty in staying up to date with developments. It was noted that attending conferences and industry workshops were very beneficial in assisting professionals to stay up to date. 
`Things are moving so quickly that to rely on books which have taken two years to produce [does not suffice, because] things have moved on. Books (plus journals) used to be the main reference. Now things are moving at such a rapid pace.'
A range of techniques were used to qualify whether specific lighting was `safe' or not. The most common practical technique was spot metering of lux values incident on specific objects, and selective dimming to drop incident lighting to the desired lux level in response to this. Some larger institutions with access to microfading equipment were able to use this in the determination of sensitivity of specific objects. One interviewee provided details of a spectral power distribution based method for considering the safety specifically of phosphor based LED illumination, whereby the height of the blue peak was compared to the height of the peak of the broader peak above 500nm, and if the former was more than three times the height of the latter, that lighting was singled out as potentially not safe. Other interviewees had heard of this criteria, and some used it as a rough guide, but one remarked that it was `fairly arbitrary'. One of the most succinct and perhaps astute responses was `what is safe?'. One interviewee referred to the website of Padfield70 where considerations for the `RE%' (`relative spectral sensitivity normalised exposure values'), using the damage functions described in 71 are used.
The interviewees generally did not consider characteristics of the displayed objects such as geometry of illumination, broadly considering this to be the remit of a lighting designer or exhibition designer rather than a conservator.
One of the most interesting, and perhaps surprising findings was the ubiquity of visual testing of lighting, generally performed prior to any large installation. For relamping, manufacturer supplied attribute values were generally relied upon as this required less time/effort and was cheaper. The most common procedure for new installations (regards quoted figures vs. visual testing) seemed to be for an informal and minimal visual test to occur before widespread installation. In a single case however, visual testing was actively avoided (on the basis that visual testing could not deliver meaningful insights where the aim was accurate rendering as opposed to visually pleasing rendering) and in another case large scale visual testing was performed, including many different types/brands of lamp and a large number of museum staff, and a final decision was made almost entirely on the results of this testing.
LEDs are now used, at least in part, in all the institutions involved in this survey. In several they are the primary lighting technology but in a small number they are used sparingly, only in applications such as the lighting of text information panels. In one they are used in a particularly minimal fashion, although this was attributed to the fact that the museum is moving location in the near future and thus capital investments in building infrastructure were being avoided for the present time. There was no one specific brand or type which seemed to be ubiquitous across institutions, rather each institution appeared to have relationships with different manufacturers and suppliers.
Most interviewees were aware of warnings which had been issued and well publicised in the mainstream press72 regards the potential of LED sources to be especially degrading for specific objects. Interviewees saw these warnings as controversial and likely unwarranted, and were confident that research had been conducted which cleared LEDs of causing an unacceptable level of damage in comparison with alternative technologies. When asked how they might assess a light source for safety, most replied that safety was assessed solely through use of an illuminance meter and lux targets, not through analysis of the \gls{SPD} or any other lighting attribute. Those who did critically assess the \gls{SPD} generally used no specific tools to do so, focusing attention on the wavelength of the spectral emission peak.
`We never normally adjust the lighting type for a given artwork, we adjust the intensity'
`All lighting we measure the \gls{SPD}, and check it is reasonable'
They key driver behind the adoption of LEDs appears to be energy efficiency increases and energy use reductions, as required by institution wide directives, or as part of applications for planning permission. Secondary to this consideration, benefits noted include: decreased maintenance costs from extended lifetime of products, and a lack of availability of traditional bulbs, sometimes due to specific legislation which has in effect phased out some traditional technologies. One element holding back some interviewees from further investment was a residual feeling that this new technology was not yet fully proven. Many pointed out that the claims made regards extended lifetime of LEDs were yet to be proven in real world environments due to the relatively new nature of the technology. Some also noted the high costs associated with having to change the underlying lighting infrastructure, where retrofitting wasn't possible or appropriate. A final note on this section – some interviewees were unsure about the ability of LEDs to remain colour stable over the long expected lifetime of the products.
Interviewees reported that visitors had not generally responded to any changes in lighting technology, and this was taken to mean that any switch to LED had not been negatively received. It was inconclusive whether or not the technology was well received however. This could be a meaningful avenue for future work. In terms of the professionals own opinions of LED lighting, all seemed favourable, though it was unclear how much of this effect was caused by extraneous or related effects such as a placebo effect due to the excitement of new technology, or different chromaticities or brightnesses of replacement technologies.  
I like what I've seen. The galleries where we have just LED spots, I feel happier. I went to [another institution, with abundant LED lighting], I really like the galleries where they had LED lighting, and it was more of a gut feeling rather than something which I could put my finger on, but actually, it felt cleaner to me.
Whilst all interviewees were interested in the idea of making objects appear brighter whilst reducing the level of damage caused, the point was made that whilst spectral tuning might benefit a prototypical object, it will not necessarily benefit real objects in real environments. It was also noted that the use of lux in making decisions was particularly problematic here, where varying the location of a blue peak could easily increase the level of damage but reduce the lux value.
Most interviewees had a basic understanding of colour temperature, and a limited understanding of chromatic adaptation. The colour temperature of lighting was generally not seen as a conservation issue (though there were exceptions to this), and rather as a creative consideration. One interviewee noted that it was manipulated to great effect by external lighting designers in order to create specific effects or atmosphere.
The justification for \gls{CCT} specification values generally appears to stem from two routes. Firstly, from guidance documents such as those provided by The Getty and the GCI6, and secondly from a desire to match existing lighting, either daylight or more commonly tungsten (at around 3000K). It was rarely considered as a means to control damage, or as a way to enhance visual appearance, by those interviewed. One interviewee referred to the work of Kruithof60 and Scuello et al.63,64 as justification for choosing low \gls{CCT} illumination similar to tungsten. No specific issues relating to colour temperature were raised by interviewees.
The interviewees were very interested in the subject of accurate representation, and almost all seemed to regard accurate object representation as a key priority. The figures for Ra quoted in internal documents at each institution were either 80, 85 or 90, as a minimum figure. In one institution, an Rw value was calculated for each proposed light source (the R value of the TCS with the colour shift of greatest magnitude) and a cut-off of 80 imposed. However, most interviewees seemed unsure of the practical relevance of Ra, with many considering it a rough guideline which would be considered secondarily to a visual inspection of lighting.
Those who were particularly interested in the subject gave the impression that whilst the subject was considered philosophically in great detail, the tool which was actually used to analyse the colour rendering of a light source was still generally just Ra. A few interviewees were aware of TM-30-15, and whilst it was respected, it was questioned whether it represented a real improvement over Ra.
On the subject of lighting philosophy, the opinions encountered generally aligned with the mechanics of Ra. That is to say; given a choice between illuminating an object such that it was beautified, visually restored (to a previous condition) or simply presented as it would appear under daylight/tungsten illumination, most opted for the final option. Whilst there was clear interest in the other options, and other creative ways in which to consider colour rendering, it was generally believed that the role of the museum should be to represent objects in an un-biased manner, and thus a fidelity index was an appropriate tool for discussing a light source's colour rendering properties. In the case where specialist lighting was used to special effect, the opinion was noted that 'you have to be very clear about what you are doing and why' in order to maintain the reputation of the museum as an arena for honest and unbiased representation.
Generally, the interviewees believed that visitor requirements were being met (although there is often difficulty in defining exactly what visitor requirements actually are) and no specific tool or technology was proposed that would provide a clear benefit. Several interviewees mentioned that a way to improve the accessibility of colour rendering indices (in terms of the ease with which they could be understood and applied) would be appreciated.
No interviewees knew of recent surveys similar to the present one.



\subsection{Colour Rendering Indices in Museums}

Are fidelity indices suitable for museum use? 
The first chapter of Cuttle's book `Light for Art's Sake: Lighting for Artworks and Museum Displays'19 is titled `A philosophy for the presentation of art'. The choice of title here is apt but may surprise some readers, sounding more whimsical than might be expected of a serious subject attended by scientists and engineers. The reason it is so apt is because lighting is unavoidably a creative intervention (I owe this phrase to Katherine Curran); which is to say that there is no lighting which is truly impartial, and no lighting which is truly correct in the sense of being unequivocally superior to another. All lighting decisions require choices to be made, and whilst these choices can be wrapped up to appear as an optimization problem where proximity to a particular solution is the goal, the problem always rests on the bedrock of a philosophy based decision.
In the above mentioned chapter Cuttle lays out a total of seven distinct philosophical propositions, which he poses for consideration as approaches to museum lighting, some contradictory and some with the potential to overlap:
1.	`To make the artwork appear as it would have appeared to the artist at the time of its creation
2.	To ensure that no damage due to light exposure will occur
3.	To achieve the best possible appearance of the artwork
4.	To provide optimum conditions for viewing art
5.	To impart a sense of having seen `the real thing'
6.	To assist viewers to understand the displayed objects and their reason for being there
7.	For the lighting designer to establish a distinct and recognisable style'19 
These propositions refer to museum lighting holistically, considering all aspects of museum lighting, but can be readily focused on the problem of colour appearance specifically. Before we narrow our gaze however, it is worth briefly considering a wide view of lighting attributes which may aid in the realisation of `good quality' museum lighting. Consider Scott Rosenfeld's list of the five `controllable qualities' in museum lighting; `intensity, movement [temporal artefacts], angle [modelling, avoiding glare and reflections], distribution [ambient lighting vs. spot lighting], color'20.
Whilst Cuttle's propositions make for interesting discussions and enjoyable extended pondering, they are of limited assistance in the practical task of actually specifying lighting. Thankfully, a range of tools exist for the examination of the colour rendering properties of a light source, in the form of indices which aim to numerically describe an illumination's effect on colour appearance of the objects of which it is tasked with illuminating.
Traditionally, colour rendering indices aim to offer a standardized method for calculating the colour differences induced by the substitution of a reference illuminant with a specific test source, and for comparing the relative merit of different test light sources on their ability to induce minimum change. In modern parlance this type of index should be referred to as a colour fidelity index, that is- one which is conceptually concerned with colorimetric reproduction. The term `colour rendering' has come to encompass much more than just fidelity.
Diametrically opposed in some ways to fidelity are the indices which aim to quantify `preference'. In the simplest case, a preference index will aim to provide a value that is predictive of how an observer would rate a light source it against other light sources. 
Within, and on the spectrum between these two groups there exist a range of different indices with subtly different aims and mechanisms for achieving these aims. For a thorough overview, albeit one which omits some more recent developments, see Guo and Houser21.
In practice, both of these philosophical approaches are mandated in current lighting guidance. For the most part, advice for lighting specification6,7 on the subject of colour rendering can be simplified to read `use lamps of above CRI 80 (referring tacitly to CIE Ra) but always test them visually before you buy in bulk'. Whilst this may at first seem like sensible advice, upon further inspection it in fact represents a serious contradiction, unless considered with heavy caveats. The problem rests in the fact that CIE Ra is a fidelity index, whereas any visual inspection is likely to be performed by observing the appearance of an object under the test illumination without a reference. Fidelity aims to describe accuracy of reproduction, but this is a quality which is arguably not testable by visual inspection. This contradiction seems to perpetuate unnoticed in museum practice, with lighting specifiers often abstractly declaring to target a faithful/accurate/honest/impartial representation of objects, but practically choosing light sources based on visual inspection where preference is the only criteria. There is of course a range of approaches, ranging from pure reliance on indices to almost entire reliance on visual testing.
In conclusion, no one metric exists that would satisfy the divergent aims and philosophies of museum lighting. Several distinct types of colour rendering index exist, but the existing range fall into the broad categories of `fidelity' or `preference', with the latter being particularly poorly definable due to its variability in different environments, with different user functional requirements and different intrinsic preferences between different observers. Progress could be made by breaking these broad categories into smaller more manageable specific objectives.

\subsubsection{CCT in museums}

\acrfull{CCT} describes the chromaticity of a light source on a one dimensional axis roughly from blue to yellow, based on the apparent chromaticity of a black body radiator at differing temperatures. It is often described as an important variable in museum lighting but definite recommendations, in the rare cases that they are given, are generally based on very general rules that predict that damage potential will be decreased if \gls{CCT} is minimised59, nostalgia for the appearance of tungsten lighting or questionable research on human preference60. Whilst some research appears to have found optimal colour temperatures for viewing artwork61,62,63,64,65, results often have large inter-observer variability, context dependency and it is not uncommon for the headline findings of separate studies to be in contradiction. 

Specifications often refer implicitly or explicitly to the findings of Kruithof60, who found that at lower levels of illumination, lower colour temperatures were preferred, and that at higher levels of illumination higher colour temperatures were preferred. Whilst this general trend seems to have empirical support, I suspect that there may be lighting technology based confounds, and recent researchers66 (including a meta-study of multiple other examinations67) found there to be no provable correlation between preference and colour temperature.

The CIE however, published a report showing the varying the \gls{CCT} of museum lighting could have a profound impact on the potential damage undergone by museum objects. 

%fig

This research relies heavily upon the applicability of damage functions to the specific materials in question. Many (my supervisors) think that damage functions are no good because they're too general, but I would be cautious about throwing the baby out with the bathwater.

\begin{enumerate}
    \item They're the best that we have.
    \item We need something with general applicability anyway - I don't see anyone providing tailored lighting for every object which exists.
    \item The physics checks out.
    \item They're definitely applicable to at least some objects.
\end{enumerate}

To verify the findings of the CIE, to extend their findings to \glspl{LED}, and to provide code for others to do similar research or even test their own lights/materials, I wrote some code which would calculate a \gls{DI} in \gls{MATLAB}:

% \begin{lstlisting}
% function DI = CalcDI(SPD,S_SPD)

% % Function to calculate damage index, as defined by "CIE 157:2004 Control of damage to museum objects by optical radiation"
% % Currently required PsychToolbox for SToWls, T_CIE_Y2.mat, SplineCmf

% %%

% % Set damage function
% b = 0.0115;
% S_dm_rel = exp(-b*(SToWls(S_SPD)-300)); % eq 2.5 from CIE 157:2004

% % Set photopic luminance function
% load T_CIE_Y2.mat T_CIE_Y2 S_CIE_Y2 %2 degree observer
% if S_CIE_Y2 ~= S_SPD
%     T_CIE_Y2 = SplineCmf(S_CIE_Y2,T_CIE_Y2,S_SPD); %set to same wavelength range and sampling interval
%     S_CIE_Y2 = S_SPD;
% end

% %% Compute DI

% F_dm_rel = SPD' * S_dm_rel;  %Eq. 1.3 from CIE 157:2004
% F_v_rel  = SPD' * T_CIE_Y2'; %Eq. 1.4 from CIE 157:2004
% DI = F_dm_rel./F_v_rel;      %Eq. 1.5 from CIE 157:2004

% %% Normalise

% % Load reference illuminant (CIE Illuminant A)
% load spd_CIEA.mat spd_CIEA S_CIEA
% if S_CIEA ~= S_CIE_Y2
%     spd_CIEA = SplineCmf(S_CIEA,spd_CIEA,S_CIE_Y2); %set to same wavelength range and sampling interval
%     S_CIEA = S_CIE_Y2;
% end

% % Calc DI for reference illuminant
% F_dm_rel_ref = spd_CIEA' * S_dm_rel;  %Eq. 1.3 from CIE 157:2004
% F_v_rel_ref  = spd_CIEA' * T_CIE_Y2'; %Eq. 1.4 from CIE 157:2004
% DI_ref = F_dm_rel_ref./F_v_rel_ref;   %Eq. 1.5 from CIE 157:2004

% % Normalise by DI_ref
% DI = DI/DI_ref;
% \end{lstlisting}

\footnote{\url{https://github.com/da5nsy/DamageIndex/blob/c7851e27ca1b0915013d8723db04704b49b4085e/CalcDI.m}}
% Need to work out where to put this

% Figure


\subsection{Balancing conservation with observation}
\subsection{Damage factors}
\subsection{LEDs in museums}
%\subsection{New opportunities with solid state lighting}


\section{Intrinsically Photosensitive Retinal Ganglion Cells}
Retinal rod and cone cells (of three types, l/m/s) are well established as the primary receptors for human vision, and their connections and properties are relatively well understood. Two distinct signalling pathways originate from the retina, one which is associated with image formation and the other which is considered to be non-image-forming (NIF), and which influences systems such as circadian rhythm entrancement, pupillary reflex and melatonin release. It was originally thought that rods and cones were the sole inputs to both of these pathways22.

\Glspl{RGC} combine signals from groups of cones and rods and relay these signals via the optic nerve to the lateral geniculate nucleus, which in turn processes and relays them further to the cortex for additional processing, allowing for classical vision of objects, movement and colour. \Glspl{ipRGC} are a sub-class of retinal ganglion cells, which in addition to combining and relaying signals exhibit some intrinsic photosensitivity of their own. 

This intrinsic photosenstivity was only confirmed recently23 relative to our knowledge of other retinal cell types, following a search for a retinal cell type or combination of cell types which would fit the spectral sensitivity properties found to influence entraining the circadian rhythm in humans and other animals24,25, which was dissimilar to all of the spectral sensitivities of the cell classes known at the time.

Additionally, it was found that animals and humans with no functioning cones or rods were still able to have a correctly functioning circadian system26,27, further suggesting that that the circadian rhythm was influenced by a novel receptor with a distinct photoreceptor. It is now believed to be these cells which provide input to the second of the signalling pathways originating in the retina, that which controls the NIF response to illumination.

\Glspl{ipRGC} were found to express a photopigment fitting such attributes, named melanopsin. The spectral sensitivity of melanopsin is around 480nm23,30,31,32,33 which places it between the s-cone (cyanolabe photopsin) and rod cell (rhodopic rhodopsin) peak spectral sensitivities, see Figure 9. In humans, preceptral filtering leads to a functional peak sensitivity of closer to 490nm29. There is uncertainty regarding the potential bistable nature of melanopsin, and whether this would have a functional effect on the spectral sensitivity of melanopsin. See29, page 12 for further detail on this subject.

\Glspl{ipRGC} are sparse in the retina; discounting input from other cell types, they operate at a much lower resolution that as would be required for meaningful spatial vision. They also operate much more slowly compared to other cell types, taking several seconds to respond, but are able to sustain a response in contrast to other retinal cell types which are able to respond quickly but only for short periods.
It has been proposed that \glspl{ipRGC} may allow an observer to sense a level of absolute irradiance34. This is clearly important for entrainment of the circadian rhythm, but could also have profound effects for image forming vision. Studies are now beginning to explore this influence 27,34 and the functional diversity within the relatively small group of \glspl{ipRGC} is becoming apparent35,36. 



\section{Research questions and hypotheses}
