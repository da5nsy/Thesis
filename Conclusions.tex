\chapter{General Conclusions}
\label{chap:Conclusions}

\begin{enumerate}
\item Assuming the applicability of general damage functions, potential damage may be reduced by using light sources with lower \glspl{CCT} (Figure \ref{fig:CCTvsDI}).
\item Museum professionals currently do not employ this technique, in part due to unclear results as to how \gls{CCT} interacts with observer preference (Section \ref{sec:CCTmus}).
\item One claimed cause for disagreement between previous results (involvement of melanopsin in CC) is examined with results showing no evidence for a strong or simple interaction
\item To address limitations in the performed experiments, and assist with designing future experiments, a computational study was performed to identify whether a melanopsin interaction would be beneficial for cc
\item In response to this, and observations from psycho physical work, future work is proposed
\item A novel method for performing cc experiments with a tablet computer is proposed.
\end{enumerate}

Does melanopsin play a role?
Next experiment

\section{Contribution to vision science}
\section{Recommendations for museum lighting}
\section{Future work}

