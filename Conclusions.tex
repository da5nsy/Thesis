\chapter{General Conclusions}
\label{chap:Conclusions}

\section{Summary of Conclusions}

\begin{enumerate}
\item Assuming the applicability of general damage functions, potential damage may be reduced by using light sources with lower \glspl{CCT} (Figure \ref{fig:CCTvsDI}).
\item Museum professionals currently do not employ this technique, in part due to conflicting results as to how \gls{CCT} interacts with observer preference (Section \ref{sec:CCTmus}).
\item One potential cause for these conflicting results (involvement of melanopsin in colour constancy) was examined with results showing no evidence for a strong or simple interaction (Chapters \ref{chap:LargeSphere} and \ref{chap:SmallSphere}).
\item To address limitations in the performed experiments, and assist with designing future experiments, a computational study was performed to explore whether a melanopsin interaction would be beneficial for colour constancy (Chapter \ref{chap:Melcomp}).
\item In response to this, and observations from psychophysical work, future work is proposed (Section \ref{sec:fut}).
\item A novel method for performing colour constancy experiments with a tablet computer has been provided, and recommendations for further development are suggested (Chapter \ref{chap:Tablet}).
\end{enumerate}

\section{Contributions to Museum Lighting}

The ideal approach would be to find the damage function for the specific material which is of concern, and the \gls{SPD} of the various illuminants under consideration, and compute a range of bespoke damage index values for those specific illuminants and that specific material. This could be achieved using the code provided at \url{https://github.com/da5nsy/DamageIndex}.

However, where this seems like an unreasonable or unpractical undertaking, based on computations performed here (and relying on the applicability of general damage functions) it seems that it would be a wise choice to pick museum lighting of a lower \gls{CCT} when faced with a choice between two otherwise similar light sources.

It is likely that some \glspl{CCT} will be preferred over others. However, it seems likely that luminance and colour rendering are more important factors to visual experience. With this in mind, using the choice of \gls{CCT} as a means to limit damage seems wise.

\section{Contributions to Vision Science}

The psychophysical experiments reported here set out to answer the question `Are \glspl{ipRGC} involved in colour constancy?'. These experiments do not provide evidence of a melanopic input to colour constancy. However, it is noted that these experiments explore only a small fraction of the stimulus space that \glspl{ipRGC} may act over. It is further noted that these experiments, and other experiments done in this field elsewhere, do not necessarily correspond to the conditions under which one may expect to see an effect. The computational study reported in Chapter \ref{chap:Melcomp} aimed to better define what these conditions are most likely to be.

The computational study makes the prediction that colour constancy using a melanopsin-based mechanism should be effective even when there is only one surface present in the scene. Other candidate mechanisms do not share this prediction, and it seems as though this should be a testable hypothesis.

However, since based on the results of \citet{kraft_mechanisms_1999} it seems likely that colour constancy is achieved through the interplay of a great many factors, and there is not a single panacea to the problem of illuminant variance, special care should be taken. It is likely that in some situations certain mechanisms or cues are given weight, where in another situation, others are prioritised.

\section{Future work} \label{sec:fut}

Future work relevant to the computational study and the tablet method is listed at the end of the respective chapters. A number of recommendations for future sphere-type psychophysical investigations are provided below.

\begin{itemize}
\item The experiments reported here were of relatively low luminance, and personal communications with other researchers in the field have suggested that melanopsin is more likely to be active only at relatively high luminances.
\item The experiments reported here relied upon the assumption of cross-retinal adaptation. Whilst this is an interesting area to explore, and certainly worthy of further investigation, a simpler place to start would be to use corresponding locations for adaptation and testing. For example, a screen or other illumination source comprised of a number of concentric circles of different diameters, where the centre and each annulus could be uniquely addressed, would allow an investigator to adapt a particular eccentricity of the retina, and test on that same part of the retina (or a different area). This would have the additional benefit of allowing for observer metameric matches which satisfied the various bands of different spectral sensitivity as eccentricity increases.
\item A valuable concept introduced by \citet{spitschan_selective_2015} is that of `splatter'. This is ``the expected amount of contrast on nominally silenced photoreceptor
classes for a given modulation around a given background'' and is introduced in recognition of the limitations of stimulus control. Including a control condition, or multiple control conditions, where a conservative estimate of the amount of cone/rod contrast which is likely to have avoided silencing is presented, can allow for an effect due to the target variable to be distinguished from noise resulting from the limits of stimulus control. 
\item It is recommended to avoid manual input from observers wherever possible, as this seems to be a considerable source of error. \citet{allen_form_2019} introduced a method for performing perceptual nulling in this type of experiment using an alternative forced choice paradigm, where potential metameric pairs were presented as spatial sinusoidal gratings of one of 4 orientations. An observer had to report the orientation, and the orientations where performance fell to chance levels were selected as `functionally metameric' pairs for each observer. An analogous method to replace the task of achromatic setting might be the method of \citet{smithson_colour_2004}, whereby observers make category judgements rather than tuning the appearance of a stimulus. This has the advantage that stimuli can be shown for a defined amount of time, avoiding the problem whereby the test stimulus affects the observer's state of adaptation. Informal discussion with these investigators has suggested that whilst this method garners data with low levels of noise, it is relatively time-consuming.
\item It does not appear as though this type of experiment is susceptible to the issues identified by \citet{spitschan_selective_2015} whereby cones in the shadow of retinal blood vessels have a different effective spectral sensitivity, and thereby break designed metamerism. This seems only to affect high temporal frequency presentations, which suggests that the perceptual nulling stage may be at risk. \citet{zele_melanopsin_2018} propose a method to avoid this issue by desensitising cones with temporal white noise which is nominally melanopsin silent.
\item In experiments which explore long term adaptation to multiple nominally metameric conditions it would be beneficial to test throughout the experiment whether metamerism was holding. This could be achieved, for example, with brief repeats of the perceptual nulling task, interleaved into the main task.
\item The computational study performed in Chapter \ref{chap:Melcomp} identified that an absolute melanopic signal (`first-level', to use the terminology used in that chapter) may be an inferior target signal compared to a signal which was in some way normalised. The specific normalisation identified in the computational study was a normalisation by luminance, following the template for computing \gls{MB} chromaticity values, though further research is advised before settling on this as a new target variable (it is quite plausible that normalisation to another signal might provide a superior signal).
\item A final note, which is in recognition of a change to standard practice which has occurred during recent years: future experiments of this type (so long as there is a clear model from which a definite prediction can be made) should pre-register the hypothesis and methodology, since it seems likely that this with this type of experiment (with many uncontrolled / uncontrollable factors and high levels of noise) may be particularly susceptible to post-hoc analysis.
\end{itemize}





