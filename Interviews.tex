\chapter{Interviews with Museum Professionals}
\label{chap:Interviews}

\textit{The following work has been presented as an oral presentation at SEAHA 2016 \citep{garside_interviewing_2016}, and published as a journal article: `How is museum lighting selected? An insight into current practice in UK museums' \citep{garside_how_2017}. A short summary of that article is presented here.}

\medskip


\section{Introduction}
In order to gain an understanding of how museum professionals currently specify lighting, interviews were conducted with 12 museum professionals representing 10 UK based museums, galleries or historic property management groups, in spring of 2016. 

The goal of this work was to understand how lighting decisions were being made at this time, in order to increase our understanding of the decisions taken within practice, and also to identify gaps which would benefit from research development.

\section{Methodology}
The interviews were semi-structured around a set of questions composed by the author and their academic supervisors (reproduced in \citet{garside_how_2017}). The choice to conduct these interviews in a semi-structured format as opposed to a fully-structured one was based on the desire to allow unanticipated topics to enter the conversation, so as to limit the potential for important subjects to be neglected due to naivety of foresight. The conversational format of the interviews meant that the resulting data was qualitative rather than quantitative, which to an extent hinders comparison, but it was decided that the variety of job roles and institutional sizes, and the small sample size, already made quantitative comparison of limited meaningfulness for this investigation. Interviewees were granted anonymity, on the basis that this would allow them to answer questions more freely.

The interviewees were contacted through introductions from supervisors, personal connections, or cold emails. A small number represented UK-wide groups, but the majority represented London based institutions. Almost all held conservation based roles (the exception being an `exhibitions manager') and noted that their principal responsibility regards lighting was controlling the safety of lighting regards its ability to cause damage to the objects held and displayed by their respective institutions. This generally involved monitoring and analysis of existing lighting systems and natural illumination, creating general guidance documents for the specification of lighting in their specific institutions and for loan items, and providing guidance and recommendations for the fitting out of new galleries or gallery refits. Few considered themselves directly responsible for the appearance of museum objects, considering this to be a creative decision outside of their remits. Future work considering the perspective of others involved in museum lighting, such as representatives from external lighting design companies, is required.

Whilst a small number considered themselves active in the area of lighting research, all interviewees were responsible for more conservation considerations aside from lighting, limiting the practicable level of specialisation. Many considered communication and dissemination a key part of their role, often noting they often found themselves in the position of needing to educate other teams within their institution on subjects including lighting. 

\begin{itquote}{}
This is not my science, my job is pulling it out and presenting it to others.
\end{itquote}

\section{Summary of Responses}

\subsection{`Good lighting'}
Asked what each considered to be `good lighting', responses included `safe', `invisible' (you don't notice it), and `lighting which is appropriate for your objects and your exhibition' (noting the variability of requirements dependent on the particular object(s) being presented). Asked to present a list or range of priorities, many focused on the safety requirements of lighting. The principal safety concern for lighting was that it fell below specific illuminance criteria, dependent on the assumed sensitivity category of the object in question. The specific target values were generally those provided by \citet{thomson_museum_1986} of 50lx for sensitive items and 200lx for less sensitive items (which are based on the visual preference work of \citet{loe_preferred_1982}).

Considering a scale of other priorities, following the requirement for appropriate lux levels, considerations included: limiting/excluding \gls{UV}, obtaining an acceptable \gls{CRI} value, and time and capital costs associating with fitting and maintenance. Energy efficiency was also a driver, but this did not seem to be a factor within technology category, rather it was noted that many institutions are switching to \gls{LED} because of increased efficiency, but that the difference in efficiency between one \gls{LED} type and another was negligible compared to the saving in contrast to lighting technology which \gls{LED} was replacing, which was often tungsten.

\subsection{Roles and Tools}
The range of roles played in the procurement of lighting varied amongst those interviewed. Whilst some created guidance documents which would then be passed on to estates teams and exhibition designers or lighting designers specifically, others had a much more hands-on role, testing specific lighting before installation or making recommendations on a case-by-case basis. It was noted that whilst relamping and retrofitting was generally handled by `in house' teams, where new galleries or large temporary exhibitions were created it was common for an external lighting design company to be contracted to perform the work. When asked whether recommendations were normally followed, the general reply was that recommendations for lux exposure and UV content were almost always followed, but other recommendations (such as for \gls{CRI} or \gls{CCT}) were more loosely interpreted. In many cases recommendations for \gls{CCT} were not made.

When asked about the tools used to make such recommendations and choices, responses included references to guidelines and reference sources (most often \citet{thomson_museum_1986}, though also \citet{druzik_guidelines_2012} and \citet{british_standards_institution_pas_2012}), references to specific units such as `lux' used in tandem with recommendations from guidelines such as above, indices such as the \gls{CIE} general colour rendering index R$_a$ (referred ubiquitously to as `\gls{CRI}' by interviewees) and notes of specific conferences which were attended in order to stay up to date with the research on the subject. There was a general feeling that the current climate was one of swift technological change in lighting, which created an increased difficulty in staying up to date with developments. It was noted that attending conferences and industry workshops were very beneficial in assisting professionals to stay up to date. 

\begin{itquote}{}
Things are moving so quickly that to rely on books which have taken two years to produce [does not suffice, because] things have moved on. Books (plus journals) used to be the main reference. Now things are moving at such a rapid pace.
\end{itquote}

\subsection{Quantifying Safety}
A range of techniques were used to qualify whether specific lighting was `safe' or not. The most common practical technique was spot metering of lux values incident on specific objects, and selective dimming to drop incident lighting to the desired lux level in response to this. Some larger institutions with access to microfading equipment (whereby an intense light source is focused onto a very small area of an object, such that damage potential can be tested but in a manner which causes minimal damage to the object as a whole) were able to use this in the determination of sensitivity of specific objects. One interviewee provided details of a spectral power distribution based method for considering the safety specifically of phosphor based \gls{LED} illumination, whereby the height of the blue peak was compared to the height of the peak of the broader peak above 500nm, and if the former was more than three times the height of the latter, that lighting was singled out as potentially not safe. Other interviewees had heard of this criteria, and some used it as a rough guide, but one remarked that it was ``fairly arbitrary''. One of the most succinct and perhaps astute responses was ``what is `safe'?''. One interviewee referred to the website of \citet{padfield_relative_2012} where considerations for the `RE\%' (`relative spectral sensitivity normalised exposure values'), using the damage functions described by \citet{aydinli_deterioration_1990} are used.

The interviewees generally did not consider characteristics of the displayed objects such as geometry of illumination, broadly considering this to be the remit of a lighting designer or exhibition designer rather than a conservator.

\subsection{Visual Testing}
One of the most interesting, and perhaps surprising findings was the ubiquity of visual testing of lighting, generally performed prior to any large installation. For relamping, manufacturer supplied attribute values were generally relied upon as this required less time/effort and was cheaper. The most common procedure for new installations seemed to be for an informal and minimal visual test to occur before widespread installation. In a single case however, visual testing was actively avoided (on the basis that visual testing could not deliver meaningful insights where the aim was accurate rendering as opposed to visually pleasing rendering) and in another case large scale visual testing was performed, including many different types/brands of lamp and a large number of museum staff, and a final decision was made almost entirely on the results of this testing.

\subsection{LED Usage}
\Glspl{LED} are now used, at least in part, in all the institutions involved in this survey. In several they are the primary lighting technology but in a small number they are used sparingly, only in applications such as the lighting of text information panels. In one they are used in a particularly minimal fashion, although this was attributed to the fact that the museum is moving location in the near future and thus capital investments in building infrastructure were being avoided for the present time. There was no one specific brand or type which seemed to be ubiquitous across institutions, rather each institution appeared to have relationships with different manufacturers and suppliers.

Most interviewees were aware of warnings which had been issued and well publicised in the mainstream press \citep{lewis_smith_will_2013} regards the potential of \gls{LED} sources to be especially degrading for specific objects. Interviewees saw these warnings as controversial and likely unwarranted, and were confident that research had been conducted which cleared \Glspl{LED} of causing an unacceptable level of damage in comparison with alternative technologies. When asked how they might assess a light source for safety, most replied that safety was assessed solely through use of an illuminance meter and lux targets, not through analysis of the \gls{SPD} or any other lighting attribute. Those who did critically assess the \gls{SPD} generally used no specific tools to do so, focusing attention on the wavelength of the spectral emission peak.

\begin{itquote}{}
We never normally adjust the lighting type for a given artwork, we adjust the intensity.
\end{itquote}

\begin{itquote}{}
For all lighting we measure the \gls{SPD}, and check it is reasonable.
\end{itquote}

They key driver behind the adoption of \Glspl{LED} appears to be energy efficiency increases and energy use reductions, as required by institution-wide directives, or as part of applications for planning permission. Secondary to this consideration, benefits noted include: decreased maintenance costs from extended lifetime of products, and a lack of availability of traditional bulbs, sometimes due to specific legislation which has in effect phased out some traditional technologies. One element holding back some interviewees from further investment was a residual feeling that this new technology was not yet fully proven. Many pointed out that the claims made regards extended lifetime of \Glspl{LED} were yet to be proven in real world environments due to the relatively new nature of the technology. Some also noted the high costs associated with having to change the underlying lighting infrastructure, where retrofitting wasn't possible or appropriate. A final note on this section - some interviewees were unsure about the ability of \Glspl{LED} to remain colour stable over the long expected lifetime of the products.

Interviewees reported that visitors had not generally responded to any changes in lighting technology, and this was taken to mean that any switch to \gls{LED} had not been negatively received. It was inconclusive whether or not the technology was positively received however. This could be a meaningful avenue for future work. In terms of the professionals own opinions of \gls{LED} lighting, all seemed favourable, though it was unclear how much of this effect was caused by extraneous or related effects such as a placebo effect due to the excitement of new technology, or different chromaticities or brightnesses of replacement technologies.  

\begin{itquote}{}
I like what I've seen. The galleries where we have just \gls{LED} spots, I feel happier. I went to [another institution, with abundant \gls{LED} lighting], I really like the galleries where they had \gls{LED} lighting, and it was more of a gut feeling rather than something which I could put my finger on, but actually, it felt cleaner to me.
\end{itquote}

\subsection{Spectral Tuning of Light Sources}
Whilst all interviewees were interested in the idea of making objects appear brighter whilst reducing the level of damage caused, the point was made that whilst spectral tuning might benefit a prototypical object, it will not necessarily benefit real objects in real environments. It was also noted that the use of lux in making decisions was particularly problematic here, where varying the location of a blue peak could easily increase the level of damage but reduce the lux value.
Most interviewees had a basic understanding of colour temperature, and a limited understanding of chromatic adaptation. The colour temperature of lighting was generally not seen as a conservation issue (though there were exceptions to this), and rather as a creative consideration. One interviewee noted that it was manipulated to great effect by external lighting designers in order to create specific effects or atmosphere.

The justification for \gls{CCT} specification values generally appears to stem from two routes. Firstly, from guidance documents such as those provided by The Getty and the Canadian Conservation Institute \citep{druzik_guidelines_2012}, and secondly from a desire to match existing lighting, either daylight or more commonly tungsten (at around 3000K). It was rarely considered as a means to control damage, or as a way to enhance visual appearance, by those interviewed. One interviewee referred to the work of \citet{kruithof_tubular_1941} and \citet{scuello_museum_2004,scuello_museum_2004-1} as justification for choosing low \gls{CCT} illumination similar to tungsten. No specific issues relating to colour temperature were raised by interviewees.

\subsection{Colour Rendering}
The interviewees were very interested in the subject of accurate representation, and almost all seemed to regard accurate object representation as a key priority. The figures for R$_a$ quoted in internal documents at each institution were either 80, 85 or 90, as a minimum figure. In one institution, an R$_w$ value was calculated for each proposed light source (the R value of the test colour sample with the colour shift of greatest magnitude, aka `worst') and a cut-off of 80 imposed. However, most interviewees seemed unsure of the practical relevance of R$_a$, with many considering it a rough guideline which would be considered secondarily to a visual inspection of lighting.

Those who were particularly interested in the subject gave the impression that whilst the subject was considered philosophically in great detail, the tool which was actually used to analyse the colour rendering of a light source was still generally just R$_a$. A few interviewees were aware of TM-30-15 \citep{ies_ies_2015}, and whilst it was respected, it was questioned whether it represented a real improvement over R$_a$.

On the subject of lighting philosophy, the opinions encountered generally aligned with the mechanics of R$_a$. That is to say; given a choice between illuminating an object such that it was beautified, visually restored (to a previous condition) or simply presented as it would appear under daylight/tungsten illumination, most opted for the final option. Whilst there was clear interest in the other options, and other creative ways in which to consider colour rendering, it was generally believed that the role of the museum should be to represent objects in an un-biased manner, and thus a fidelity index was an appropriate tool for discussing a light source's colour rendering properties. In the case where specialist lighting was used to special effect, the opinion was noted that \textit{``you have to be very clear about what you are doing and why''} in order to maintain the reputation of the museum as an arena for honest and unbiased representation.

\section{Conclusion}
Generally, the interviewees believed that visitor requirements were being met (although there is often difficulty in defining exactly what visitor requirements actually are) and no specific tool or technology was proposed that would provide a clear benefit. Several interviewees mentioned that a way to improve the accessibility of colour rendering indices (in terms of the ease with which they could be understood and applied) would be appreciated. No interviewees knew of recent surveys similar to the present one.

The impact that these interviews had on the research which followed is such; it became clear that \gls{CCT} was a tool which could be used to reduce damage, but which was not being used at the time. One of the barriers to use was a lack of understanding of how \gls{CCT} interacted with other visual appearance properties and preference. It therefore seemed valuable to attempt to extend our understanding of chromatic adaptation and colour constancy, with the hope that this would allow museum professionals to limit damage through specification of lower values of \gls{CCT}.