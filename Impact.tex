\begin{impactstatement}

The applied goal of this research was to increase our understanding of colour constancy so as to advise museums on how to reduce damage to objects without degrading visitor experience. Asking whether ipRGCs play a role in colour constancy is additionally valuable to the vision science community, and to lighting engineering applications beyond the museum world.

In recent years there has been considerable take-up of LED lighting in museums, driven predominantly by energy-efficiency gains mandated at an institutional level. However, the spectrum of LEDs differs considerably from that of other lighting technologies, and so standard methods for limiting damage may be less effective than they have previously been. Code written for this thesis, and made available online, provides a simple means for computing damage predictions for specific light sources. 

A series of interviews was performed with museum professionals to determine the current tools and methods which are in use by those making decisions regarding lighting. This information should allow lighting researchers and industry partners to better identify areas where improvement might be most beneficial, in terms of education, communication, and technological development.

There has recently been a great amount of interest in the lighting engineering community regarding the concept of human-centric lighting, which considers the impact of artificial lighting on the circadian rhythms of the inhabitants of a space. As such, the activation of ipRGCs is often maximised (at least in the morning). Knowing whether there is an additional visual effect, which would need to be taken into consideration during design, would be of great value to those designing such lighting.

Colour constancy algorithms have value outside of understanding the visual systems of humans and other organisms. They are used extensively in imaging devices, to maintain colour balance in most types of cameras, and to aid in object recognition in computational vision applications.

The work contained in this thesis, and produced as part of this doctoral program, has the potential to be of impact to the fields of museum conservation, lighting engineering, visual neuroscience, and digital imaging.  



% tablet method

% improvements to existing methodology -spheres

% lighting - advanced a hypothesis re melanopsin

% improving the museum environment

% don't forget interviews

% conference and journal pubs

% organising conferences

%% ----------- %%

% From http://www.grad.ucl.ac.uk/essinfo/docs/Impact-Statement-Guidance-Notes-for-Research-Students-and-Supervisors.pdf
%
%\begin{quote}
%The statement should describe, in no more than 500 words, how the expertise, knowledge, analysis,
%discovery or insight presented in your thesis could be put to a beneficial use. Consider benefits both
%inside and outside academia and the ways in which these benefits could be brought about.
%
%The benefits inside academia could be to the discipline and future scholarship, research methods or
%methodology, the curriculum; they might be within your research area and potentially within other
%research areas.
%
%The benefits outside academia could occur to commercial activity, social enterprise, professional
%practice, clinical use, public health, public policy design, public service delivery, laws, public
%discourse, culture, the quality of the environment or quality of life.
%
%The impact could occur locally, regionally, nationally or internationally, to individuals, communities or
%organisations and could be immediate or occur incrementally, in the context of a broader field of
%research, over many years, decades or longer.
%
%Impact could be brought about through disseminating outputs (either in scholarly journals or
%elsewhere such as specialist or mainstream media), education, public engagement, translational
%research, commercial and social enterprise activity, engaging with public policy makers and public
%service delivery practitioners, influencing ministers, collaborating with academics and non-academics
%etc.
%
%Further information including a searchable list of hundreds of examples of UCL impact outside of
%academia please see \url{https://www.ucl.ac.uk/impact/}. For thousands more examples, please see
%\url{http://results.ref.ac.uk/Results/SelectUoa}.
%\end{quote}
\end{impactstatement}