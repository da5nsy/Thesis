\documentclass{article}

\usepackage{libertine}

\title{Museum Lighting, Color Constancy and Melanopsin.}
\date{} 

\begin{document}
\maketitle

\section{Extended Abstract}

Museums seek a pragmatic compromise between lighting which causes minimal damage to objects, and lighting which allows maximal visitor enjoyment. Generally this is achieved by following industry recommendations for maximum luminance.

A complementary way to reduce damage, highlighted by the Commission Internationale de l'Eclairage in a 2004 publication, is to choose illumination of a lower Correlated Colour Temperature (CCT), since radiation of longer wavelengths has a lower potential for damage. A series of interviews with museum professionals showed that this technique is not currently being employed. Computations performed on a range of commercially available lighting products showed a strong correlation between Damage Index (DI) and CCT, and a difference of greater than a factor of two between the low and high CCT sources. 
Colorimetry assumes seamless adaptation to any chromaticity of illumination, and so switching CCT should theoretically have no impact upon visitor experience, but it is not clear whether this is the case in reality. Despite recent work on preferred CCT for museum environments, results often disagree, and there is no clear picture of what might drive such a preference, or in fact how the mechanism(s) underlying color constancy operate.

One explanation for this uncertainty might be relatively newly discovered intrinsically photosensitive retinal ganglion cells (ipRGCs). For a range of reasons it seems plausible that signals originating from these cells might play a role in chromatic adaptation, and that this role may explain the uneven results from historic experiments on color constancy and CCT preference.

Initially, multiple psychophysical experiments were attempted, with the aim of exploring the possible role of melanopic signals on colour constancy, using variations on the Achromatic Point Setting (APS) method. During this time a variant of this method which uses a computer tablet was developed and assessed. The key benefits of this method are that it can be used in naturalistic environments (in contrast to existing methods), and that is easy and quick for a naive observer to perform (meaning more observers and less exposure to knowledge bias).

Following inconclusive results from these experiments, an exploratory computational experiment was performed to assess whether a melanopsin-based signal could theoretically be beneficial to the human visual system in solving the problem of variable illumination, based on measurements of daylight and natural object reflectances.



\end{document}