% UCL Thesis LaTeX Template
%  (c) Ian Kirker, 2014
% 
% This is a template/skeleton for PhD/MPhil/MRes theses.
%
% It uses a rather split-up file structure because this tends to
%  work well for large, complex documents.
% We suggest using one file per chapter, but you may wish to use more
%  or fewer separate files than that.
% We've also separated out various bits of configuration into their
%  own files, to keep everything neat.
% Note that the \input command just streams in whatever file you give
%  it, while the \include command adds a page break, and does some
%  extra organisation to make compilation faster. Note that you can't
%  use \include inside an \include-d file.
% We suggest using \input for settings and configuration files that
%  you always want to use, and \include for each section of content.
% If you do that, it also means you can use the \includeonly statement
%  to only compile up the section you're currently interested in.
% You might also want to put figures into their own files to be \input.

% For more information on \input and \include, see:
%  http://tex.stackexchange.com/questions/246/when-should-i-use-input-vs-include


% Formatting rules for theses are here: 
%  http://www.ucl.ac.uk/current-students/research_degrees/thesis_formatting
% Binding and submitting guidelines are here:
%  http://www.ucl.ac.uk/current-students/research_degrees/thesis_binding_submission

% This package goes first and foremost, because it checks all 
%  your syntax for mistakes and some old-fashioned LaTeX commands.
% Note that normally you should load your documentclass before 
%  packages, because some packages change behaviour based on
%  your document settings.
% Also, for those confused by the RequirePackage here vs usepackage
%  elsewhere, usepackage cannot be used before the documentclass
%  command, while RequirePackage can. That's the only functional
%  difference as far as I'm aware.
\RequirePackage[l2tabu, orthodox]{nag}


% ------ Main document class specification ------
% The draft option here prevents images being inserted,
%  and adds chunky black bars to boxes that are exceeding 
%  the page width (to show that they are).
% The oneside option can optionally be replaced by twoside if
%  you intend to print double-sided. Note that this is
%  *specifically permitted* by the UCL thesis formatting
%  guidelines.
%
% Valid options in terms of type are:
%  phd
%  mres
%  mphil
%\documentclass[12pt,phd,draft,a4paper,oneside]{ucl_thesis}
\documentclass[12pt,phd,a4paper,twoside]{ucl_thesis}


% Package configuration:
%  LaTeX uses "packages" to add extra commands and features.
%  There are quite a few useful ones, so we've put them in a 
%   separate file.
% -------- Packages --------

% This package just gives you a quick way to dump in some sample text.
% You can remove it -- it's just here for the examples.
\usepackage{blindtext}

% This package means empty pages (pages with no text) won't get stuff
%  like chapter names at the top of the page. It's mostly cosmetic.
\usepackage{emptypage}

% The graphicx package adds the \includegraphics command,
%  which is your basic command for adding a picture.
\usepackage{graphicx}

% The float package improves LaTeX's handling of floats,
%  and also adds the option to *force* LaTeX to put the float
%  HERE, with the [H] option to the float environment.
\usepackage{float}

% The amsmath package enhances the various ways of including
%  maths, including adding the align environment for aligned
%  equations.
\usepackage{amsmath}

% Use these two packages together -- they define symbols
%  for e.g. units that you can use in both text and math mode.
\usepackage{gensymb}
\usepackage{textcomp}
% You may also want the units package for making little
%  fractions for unit specifications.
%\usepackage{units}


% The setspace package lets you use 1.5-sized or double line spacing.
\usepackage{setspace}
\setstretch{1.5}

% That just does body text -- if you want to expand *everything*,
%  including footnotes and tables, use this instead:
%\renewcommand{\baselinestretch}{1.5}


% PGFPlots is either a really clunky or really good way to add graphs
%  into your document, depending on your point of view.
% There's waaaaay too much information on using this to cover here,
%  so, you might want to start here:
%   http://pgfplots.sourceforge.net/
%  or here:
%   http://pgfplots.sourceforge.net/pgfplots.pdf
%\usepackage{pgfplots}
%\pgfplotsset{compat=1.3} % <- this fixed axis labels in the version I was using

% PGFPlotsTable can help you make tables a little more easily than
%  usual in LaTeX.
% If you're going to have to paste data in a lot, I'd suggest using it.
%  You might want to start with the manual, here:
%  http://pgfplots.sourceforge.net/pgfplotstable.pdf
%\usepackage{pgfplotstable}

% These settings are also recommended for using with pgfplotstable.
%\pgfplotstableset{
%	% these columns/<colname>/.style={<options>} things define a style
%	% which applies to <colname> only.
%	empty cells with={--}, % replace empty cells with '--'
%	every head row/.style={before row=\toprule,after row=\midrule},
%	every last row/.style={after row=\bottomrule}
%}


% The mhchem package provides chemistry formula typesetting commands
%  e.g. \ce{H2O}
%\usepackage[version=3]{mhchem}

% And the chemfig package gives a weird command for adding Lewis 
%  diagrams, for e.g. organic molecules
%\usepackage{chemfig}

% The linenumbers command from the lineno package adds line numbers
%  alongside your text that can be useful for discussing edits 
%  in drafts.
% Remove or comment out the command for proper versions.
%\usepackage[modulo]{lineno}
% \linenumbers 


% Alternatively, you can use the ifdraft package to let you add
%  commands that will only be used in draft versions
%\usepackage{ifdraft}

% For example, the following adds a watermark if the draft mode is on.
%\ifdraft{
%  \usepackage{draftwatermark}
%  \SetWatermarkText{\shortstack{\textsc{Draft Mode}\\ \strut \\ \strut \\ \strut}}
%  \SetWatermarkScale{0.5}
%  \SetWatermarkAngle{90}
%}


% The multirow package adds the option to make cells span 
%  rows in tables.
\usepackage{multirow}


% Subfig allows you to create figures within figures, to, for example,
%  make a single figure with 4 individually labeled and referenceable
%  sub-figures.
% It's quite fiddly to use, so check the documentation.
%\usepackage{subfig}

% The natbib package allows book-type citations commonly used in
%  longer works, and less commonly in science articles (IME).
% e.g. (Saucer et al., 1993) rather than [1]
% More details are here: http://merkel.zoneo.net/Latex/natbib.php
%\usepackage{natbib}

% The bibentry package (along with the \nobibliography* command)
%  allows putting full reference lines inline.
%  See: 
%   http://tex.stackexchange.com/questions/2905/how-can-i-list-references-from-bibtex-file-in-line-with-commentary
\usepackage{bibentry} 

% The isorot package allows you to put things sideways 
%  (or indeed, at any angle) on a page.
% This can be useful for wide graphs or other figures.
%\usepackage{isorot}

% The caption package adds more options for caption formatting.
% This set-up makes hanging labels, makes the caption text smaller
%  than the body text, and makes the label bold.
% Highly recommended.
\usepackage[format=hang,font=small,labelfont=bf]{caption}

% If you're getting into defining your own commands, you might want
%  to check out the etoolbox package -- it defines a few commands
%  that can make it easier to make commands robust.
\usepackage{etoolbox}

% Loads a scaled version of the Helvetica font
\usepackage{helvet}
% Sets the sans-serif family as the default family
\renewcommand{\familydefault}{\sfdefault}




% Sets up links within your document, for e.g. contents page entries
%  and references, and also PDF metadata.
% You should edit this!
%%
%% This file uses the hyperref package to make your thesis have metadata embedded in the PDF, 
%%  and also adds links to be able to click on references and contents page entries to go to 
%%  the pages.
%%

% Some hacks are necessary to make bibentry and hyperref play nicely.
% See: http://tex.stackexchange.com/questions/65348/clash-between-bibentry-and-hyperref-with-bibstyle-elsart-harv
% See also: hyperref documentation section on bibentry, 9.1.7, p39
% This is the first half of the bibentry/hyperref hack
% The second half is just after begin{document} in Main.tex
%\makeatletter
%    \let\saved@bibitem\@bibitem
%\makeatother
%\usepackage{bibentry}

% v-- this setup can be useful for checking links
%\usepackage[bookmarks,colorlinks=true,backref,,hyperindex,debug,pdftex]{hyperref}
% v-- and this is the serious setup

\usepackage[bookmarks,pagebackref,pdftex,hidelinks]{hyperref} 
\renewcommand*\backref[1]{\ifx#1\relax \else (page #1) \fi}

\AtBeginDocument{
    \hypersetup{
        anchorcolor=purple,
        citecolor=green,
        pdfsubject={A thesis upon the subject of Museum Lighting, Colour Constancy and Melanopsin.},
        pdfkeywords={Museum Lighting, Colour Constancy, Chromatic Adaptation, Melanopsin},
        pdfauthor={Daniel Garside},
        pdftitle={Museum Lighting, Colour Constancy and Melanopsin},
    }
}

% Creates hyperlinks automatically from DOIs, and allows for use of \doi{} rather than \url{} which means it can handle non-allowed characters in DOIs.
\usepackage{doi}

% uses standard font for URLs
\urlstyle{same}


% And then some settings in separate files.
% These settings are from:
%  http://mintaka.sdsu.edu/GF/bibliog/latex/floats.html

% They give LaTeX more options on where to put your figures, and may
%  mean that fewer of your figures end up at the tops of pages far
%  away from the thing they're related to.

% Alters some LaTeX defaults for better treatment of figures:
% See p.105 of "TeX Unbound" for suggested values.
% See pp. 199-200 of Lamport's "LaTeX" book for details.

%   General parameters, for ALL pages:
\renewcommand{\topfraction}{1}	% max fraction of floats at top
\renewcommand{\bottomfraction}{1}	% max fraction of floats at bottom

%   Parameters for TEXT pages (not float pages):
\setcounter{topnumber}{2}
\setcounter{bottomnumber}{2}
\setcounter{totalnumber}{4}     % 2 may work better
\setcounter{dbltopnumber}{2}    % for 2-column pages
\renewcommand{\dbltopfraction}{0.9}	% fit big float above 2-col. text
\renewcommand{\textfraction}{0.2}	% allow minimal text w. figs (previously 0.07, DG change to 0.2 based on https://tex.stackexchange.com/questions/192622/is-there-a-value-for-textfraction-and-totalnumber-floatpagefraction-that-i-s)

%   Parameters for FLOAT pages (not text pages):
\renewcommand{\floatpagefraction}{0.7}	% require fuller float pages
% N.B.: floatpagefraction MUST be less than topfraction !!
\renewcommand{\dblfloatpagefraction}{0.7}	% require fuller float pages

% remember to use [htp] or [htpb] for placement,
% e.g. 
%  \begin{figure}[htp]
%   ...
%  \end{figure} % For things like figures and tables
\bibliographystyle{plainnat}   % For bibliographies

% These control how many number sections your subsections will take
%    e.g. Section 2.3.1.5.6.3
%  and how many of those will get put into the contents pages.
\setcounter{secnumdepth}{3}
\setcounter{tocdepth}{3}


\begin{document}

\nobibliography*
% ^-- This is a dumb trick that works with the bibentry package to let
%  you put bibliography entries whereever you like.
% I used this to put references to papers a chapter's work was 
%  published in at the end of that chapter.
% For more information, see: http://stefaanlippens.net/bibentry

% If you haven't finished making your full BibTex file yet, you
%  might find this useful -- it'll just replace all your
%  citations with little superscript notes.
% Uncomment to use.
%\renewcommand{\cite}[1]{\emph{\textsuperscript{[#1]}}}

% At last, content! Remember filenames are case-sensitive and 
%  *must not* include spaces.
% I may change the way this is done in a future version, 
%  but given that some people needed it, if you need a different degree title 
%  (e.g. Master of Science, Master in Science, Master of Arts, etc)
%  uncomment the following 3 lines and set as appropriate (this *has* to be before \maketitle)
% \makeatletter
% \renewcommand {\@degree@string} {Master of Things}
% \makeatother

\title{Museum Lighting, Colour Constancy and Melanopsin}
\author{Daniel Garside}
\department{Civil, Environmental and Geomatic Engineering}

\maketitle
\makedeclaration

\begin{abstract} % 300 word limit

Insert abstract

% Extended abstract: 

%Light causes damage to objects in museums. Museums seek a pragmatic compromise between lighting which causes minimal damage to objects, and lighting which allows maximal visitor enjoyment. Generally this is achieved by following industry recommendations for maximum illuminance and colour rendering index.
%
%A complementary way to reduce damage, highlighted by the Commission Internationale de l'Eclairage in a 2004 publication, is to choose illumination of a lower Correlated Colour Temperature (CCT), because such illumination will contain most radiation energy in the longer wavelengths, which are generally less damaging to objects. Computations performed for a range of commercially available lighting products showed a strong correlation between predicted damage and CCT, with a factor of two between the low and high CCT sources. Practically, a sensitive object might be displayed for twice as long under a low CCT illuminant than a high one, before passing a damage threshold. 
%
%A series of interviews with museum professionals showed that this technique is not currently being employed. One reason is a belief that changing the CCT in an environment will affect the visual experience and the atmosphere in the room.
%
%Models of vision suggest that a change in CCT alone should not affect visual experience; that in a space with a single type of lighting we should be able to adapt to any colour of illumination. This seems at most only partially true. It is true that we adapt reasonably well to the ambient illumination, both in terms of luminance and chromaticity, such that our perception of object colours relates primarily to object reflectance properties rather than the absolute intensities reaching our eyes. Generally, however, we do also seem to have an awareness of the properties of the illumination in a scene, and even a preference for some sources over others. Historically, experiments seeking to find an ideal preferred CCT have provided conflicting results.
%
%One possible reason for these conflicting results might be that an additional retinal mechanism is involved in chromatic adaptation. The studies in this thesis have investigated whether a cell group called the `intrinsically photosensitive Retinal Ganglion Cells' (ipRGCs) might be involved in colour constancy. This cell group has traditionally been considered as having no output to visual pathways, and so colour constancy experiments haven't controlled for ipRGC activation, but recent research has shown that ipRGCs do in fact have a limited input to visual pathways. There are a range of reasons that suggest that they may play a role in colour constancy specifically.
%
%To investigate the effect of different levels of ipRGC activation on an observer's state of colour constancy,	two lab-based psychophysical experiments were performed. The first sought to examine the effect of different wavelengths of light upon chromatic adaptation. Within a Ganzfeld viewing environment, illuminated by one of 16 different wavelengths of near-monochromatic light, observers performed an achromatic setting task, controlling the chromaticity of a display visible in the central field through a 4° circular aperture with two handheld sliders.
%
%In the second experiment the role of melanopsin in chromatic adaptation was more directly questioned. The same task was performed as in the first experiment, but the Ganzfeld was this time illuminated by one of two perceptually metameric lights with different melanopic irradiance levels. Neither of these experiments provided evidence for a simple or strong effect of ipRGC activation, though this is limited to the case where peripheral stimulation would affect central perception.
%
%Concurrently, a method has been developed for performing colour constancy experiments outside of the lab environment, which can be completed quickly by naive observers in `real-life' illumination conditions. The method uses a tablet computer, on which an isoluminant plane through CIELUV is presented, successively varying in orientation and spatial offset. The observer is tasked with selecting, by touching with a finger, an achromatic point from within each stimulus. From the recorded selections an estimate of the observer's state of chromatic adaptation is computed.
%
%Following these experiments a different approach was adopted: rather than asking what was the affect of varying ipRGC activation upon experimental subjects, instead the question was posed whether an ipRGC-based signal could hypothetically be useful for colour constancy, considering what we know of daylight variability and natural surface reflectance properties. A computational methodology was employed. It was found that the spectral sensitivity of melanopsin is optimal for providing a signal that can transform raw chromatic signals to an illuminant-independent space, and that this can be done without using any scene level assumptions such as grey-world.
%
%The applied goal of this research was to increase our understanding of colour constancy so as to advise museums on how to reduce damage to objects without degrading visitor experience. Asking whether ipRGCs play a role in colour constancy is additionally valuable to the vision science community, and to lighting engineering applications beyond the museum world.
\end{abstract}

\begin{impactstatement}

Insert Impact Statement

%	UCL theses now have to include an impact statement. \textit{(I think for REF reasons?)} The following text is the description from the guide linked from the formatting and submission website of what that involves. (Link to the guide: {\scriptsize \url{http://www.grad.ucl.ac.uk/essinfo/docs/Impact-Statement-Guidance-Notes-for-Research-Students-and-Supervisors.pdf}})
%
%\begin{quote}
%The statement should describe, in no more than 500 words, how the expertise, knowledge, analysis,
%discovery or insight presented in your thesis could be put to a beneficial use. Consider benefits both
%inside and outside academia and the ways in which these benefits could be brought about.
%
%The benefits inside academia could be to the discipline and future scholarship, research methods or
%methodology, the curriculum; they might be within your research area and potentially within other
%research areas.
%
%The benefits outside academia could occur to commercial activity, social enterprise, professional
%practice, clinical use, public health, public policy design, public service delivery, laws, public
%discourse, culture, the quality of the environment or quality of life.
%
%The impact could occur locally, regionally, nationally or internationally, to individuals, communities or
%organisations and could be immediate or occur incrementally, in the context of a broader field of
%research, over many years, decades or longer.
%
%Impact could be brought about through disseminating outputs (either in scholarly journals or
%elsewhere such as specialist or mainstream media), education, public engagement, translational
%research, commercial and social enterprise activity, engaging with public policy makers and public
%service delivery practitioners, influencing ministers, collaborating with academics and non-academics
%etc.
%
%Further information including a searchable list of hundreds of examples of UCL impact outside of
%academia please see \url{https://www.ucl.ac.uk/impact/}. For thousands more examples, please see
%\url{http://results.ref.ac.uk/Results/SelectUoa}.
%\end{quote}
\end{impactstatement}

\begin{acknowledgements}
Acknowledge all the things!

I also wish to note my appreciation and thanks to the various pieces of open source software I used in the writing of this thesis. For further details see \ref{appendixlabel3}.

\end{acknowledgements}

\setcounter{tocdepth}{2} 
% Setting this higher means you get contents entries for
%  more minor section headers.

\tableofcontents
\listoffigures
\printglossary[type=\acronymtype,style=long]
%\listoftables


\chapter{Introduction}
\label{chapterlabel1}

\section{Context}
Light causes damage to objects in museums. Museums seek a pragmatic compromise between lighting which causes minimal damage to objects, and lighting which allows maximal visitor enjoyment. Generally this is achieved by following industry recommendations for maximum illuminance and \gls{CRI}.

A complementary way to reduce damage, highlighted by the \gls{CIE} in a 2004 publication \citep{cie_cie_2004}, is to choose illumination of a lower \gls{CCT}, because such illumination will contain most radiation energy in the longer wavelengths, which are generally less damaging to objects. To verify and extend the work of the \gls{CIE}, a \gls{DI} was calculated for a range of commercially available lighting products (\textbf{Section \ref{sec:CCTmus}}). This showed a strong correlation between predicted damage and \gls{CCT}, with a factor of two between the low and high \gls{CCT} sources. Practically, a sensitive object might be displayed for twice as long under a low \gls{CCT} illuminant than a high one, before passing a particular damage threshold. 

A series of interviews with museum professionals showed that this technique is not currently being employed (\textbf{Chapter \ref{chap:Interviews}}). One reason is a belief that changing the \gls{CCT} in an environment will affect the visual experience and the atmosphere in the room.

Most models of vision suggest that a change in \gls{CCT} alone, within moderate limits, should not affect visual experience; that in a space with a single type of lighting we should be able to adapt to any colour of illumination. This seems at most only partially true. It is true that we adapt reasonably well to the ambient illumination, both in terms of luminance and chromaticity, such that our perception of object colours relates primarily to object reflectance properties rather than the absolute intensities reaching our eyes. Generally, however, we do also seem to have an awareness of the properties of the illumination in a scene, and even a preference for some sources over others. Historically, experiments seeking to find an ideal preferred \gls{CCT} have provided conflicting results.

One possible reason for these conflicting results might be that an additional retinal mechanism is involved in chromatic adaptation. The studies in this thesis have investigated whether a cell group called the \glspl{ipRGC} might be involved in colour constancy. This cell group has traditionally been considered as having no output to visual pathways, and so colour constancy experiments haven't controlled for \gls{ipRGC} activation, but recent research has shown that \glspl{ipRGC} do in fact have a limited input to visual pathways. There are a range of reasons that suggest that they may play a role in colour constancy specifically.

\section{Research Question}

The principal research question considered in this thesis is: \textbf{Are \glspl{ipRGC} involved in colour constancy?}

If they are, this may explain why previous investigations into preferred \gls{CCT} have produced conflicting results. If we can understand how they are involved, this may give us a new insight into how to vary the \gls{CCT} of museum illumination in such a way that we can limit damage without degrading visitor experience.

\section{Chapter Summaries}

To better understand how museum professionals currently make lighting decisions (what metrics they use, what the guidance is for using these metrics, what the current challenges and limitations are) a series of interviews was performed with museum lighting professionals \citep{garside_how_2017}. The responses to these interviews are summarised in \textbf{Chapter \ref{chap:Interviews}}.

To investigate the effect of different levels of \gls{ipRGC} activation on an observer's state of colour constancy, two lab-based psychophysical experiments were performed (\textbf{Chapters \ref{chap:LargeSphere} and \ref{chap:SmallSphere}}). The first sought to examine the effect of different wavelengths of light upon chromatic adaptation. Within a Ganzfeld viewing environment, illuminated by one of 16 different wavelengths of near-monochromatic light, observers performed an achromatic setting task, controlling the chromaticity of a display visible in the central field through a 4$^{\circ}$ circular aperture with two handheld sliders.

In the second experiment the role of melanopsin in chromatic adaptation was more directly questioned. The same task was performed as in the first experiment, but the Ganzfeld was this time illuminated by one of two perceptually metameric lights with different melanopic illuminance levels. Neither of these experiments provided evidence for a simple or strong effect of ipRGC activation.

Concurrently, a method has been developed for performing colour constancy experiments outside of the lab environment, which can be completed quickly by naive observers in `real-life' conditions (\textbf{Chapter \ref{chap:Tablet}}). The method uses a tablet computer, on which an isoluminant plane through CIELUV is presented, successively varying in orientation and spatial offset. The observer is tasked with selecting, by touching with a finger, an achromatic point from within each stimulus. From the recorded selections an estimate of the observer's state of chromatic adaptation is computed.

Following these experiments a different approach was adopted: rather than asking what was the affect of varying \gls{ipRGC} activation upon experimental subjects, instead the question was posed whether an \gls{ipRGC}-based signal could hypothetically be useful for colour constancy, considering what we know of daylight variability and natural surface reflectance properties (\textbf{Chapter \ref{chap:Melcomp}}). A computational methodology was employed for this exploratory study. It was found that a simple transform can be made from raw signals to an illuminant independent space, with the use of a melanopsin-based signal. Further, it was found that the spectral sensitivity of melanopsin is near-optimal for providing a signal for such a transformation, and that this can be done without using any scene level assumptions such as `grey-world'.

Chapter \ref{chap:Conclusions} provides a summary and discussion of results, and suggests routes for future work.

\section{Associated Publications}

{\parindent0pt \footnotesize


\subsection*{Journal Papers}

\textbf{Daniel Garside}, Katherine Curran, Capucine Korenberg, Lindsay MacDonald, Kees Teunissen, and Stuart Robson. How is museum lighting selected? An insight into current practice in UK museums. \textit{Journal of the Institute of Conservation}, 40(1):3-14, January 2017. \doi{10.1080/19455224.2016.1267025}

\subsection*{Conference Papers}

\textbf{Daniel Garside}. Initial Observations on Lighting Situations in The British Museum. In \textit{1st International Conference, Science and Engineering in Arts, Heritage and Archaeology}, page 58, London, UK, July 2015. \doi{10.6084/m9.figshare.4269665.v1}
\bigskip

\textbf{Daniel Garside}, Katherine Curran, Capucine Korenberg, Lindsay MacDonald, Kees Teunissen, and Stuart Robson. Interviewing Museum Professionals: How is Museum Lighting Selected? In \textit{SEAHA (Science and Engineering in Arts, Heritage and Archaeology)}, Oxford, UK, June 2016. \doi{10.6084/m9.figshare.5840952.v1}
\bigskip

\textbf{Daniel Garside}, Lindsay MacDonald, Kees Teunissen, Katherine Curran, Capucine Korenberg, and Stuart Robson. Potential uses for spectrally variable lighting in museum environments. In \textit{Progress in Colour Studies (PICS)}, page 100, London, UK, September 2016. \doi{10.6084/m9.figshare.4269659.v1}
\smallskip %This one appears to create a bigger gap for some reason

\textbf{Daniel Garside}, Lindsay MacDonald, Kees Teunissen, and Stuart Robson. Estimating Chromatic Adaptation in a Museum Environment Using a Tablet Computer. In
\textit{Proceedings of AIC 2016 Interim Meeting - Color in Urban Life: Images, Objects and Spaces}, pages 125-129, Santiago, Chile, October 2016. \doi{10.6084/m9.figshare.4269680.v2}
\bigskip

Lindsay MacDonald and \textbf{Daniel Garside}. Adapting to a Chromatic Environment In \textit{Proceedings of AIC 2016 Interim Meeting - Color in Urban Life: Images, Objects and Spaces}, pages 109-113, Santiago, Chile, October 2016.
\bigskip

\textbf{Daniel Garside} and Lindsay MacDonald. Investigations into the effect of different spectra upon the process of chromatic adaption. In \textit{Proceedings of the International Colour Vision Symposium (ICVS)}, page 35, Erlangen, Germany, August 2017.
\bigskip

\textbf{Daniel Garside}. Light art as a pedagogical tool for teaching the science of colour perception. In \textit{Abstracts from the 5th Visual Science of Art Conference (VSAC)}, page 381, Berlin, Germany, August 2017. \doi{10.1163/22134913-00002099}
\bigskip

\textbf{Daniel Garside}, Stuart Robson, Lindsay MacDonald, Katherine Curran, Kees Teunissen, and Capucine Korenberg.  A method for performing colour constancy studies using a tablet computer. In \textit{European Conference on Visual Perception (ECVP)}, Berlin, Germany, August 2017. \doi{10.6084/m9.figshare.5478493.v1}
\bigskip

M. Hess, \textbf{D. Garside}, T. Nelson, S. Robson, and T. Weyrich.  Object-Based Teaching and Learning for a Critical Assessment of Digital Technologies in Arts and Cultural Heritage. In \textit{ISPRS - International Archives of the Photogrammetry, Remote Sensing and Spatial Information Sciences, volume XLII-2-W5}, pages 349-354, Ottawa, Canada, August 2017. Copernicus GmbH. \doi{10.5194/isprs-archives-XLII-2-W5-349-2017}
\bigskip

L. W. MacDonald, \textbf{D. Garside}, and C. Teunissen. Melanopsin and Colour Vision. In \textit{Proceedings of 13th AIC Congress}, Jeju, South Korea, Oct 2017.
\bigskip

E. K. Webb, S. Robson, L. MacDonald, \textbf{D. Garside}, and R. Evans. Spectral and 3d Cultural Heritage Documentation Using a Modified Camera. In \textit{ISPRS - International Archives of the Photogrammetry, Remote Sensing and Spatial Information Sciences, volume XLII-2}, pages 1183-1190, Riva del Garda, Italy, May 2018. \doi{10.5194/isprs-archives-XLII-2-1183-2018}
\bigskip

\textbf{Daniel Garside}. Is there a role for melanopsin in chromatic adaptation? At \textit{The 16th International Symposium on the Science and Technology of Lighting (LS16)}, Sheffield, UK, June 2018.
\bigskip


\textbf{Daniel Garside}, Lindsay MacDonald, and Kees Teunissen. Does the spectral sensitivity of melanopsin in ipRGCs suggest a role in chromatic adaptation?  In \textit{Vision Sciences Society Annual Meeting Abstract, volume 18}, page 877, St Pete Beach, Florida, September 2018 \doi{10.1167/18.10.877}
\bigskip

\textbf{Daniel Garside} and Lindsay MacDonald. Assessing the effectiveness of a melanopsin-based signal for colour constancy. In \textit{Proceedings of the International Colour Vision Symposium (ICVS)}, page 65, Riga, Latvia, July 2019. 
\bigskip

\subsection*{Editorial Contributions}
A. Pokorska, P. Andrikopoulos, \textbf{D. Garside}, and C. Coon, editors. Book of Abstracts, Museum Lighting Symposium and Workshops.  \textit{1st International Museum Lighting Symposium \& Workshops}, London, UK, September 2017. \doi{10.14324/000.bk.10048078}

}

\section{Affiliations}
This project was supported by an EPSRC (Engineering and Physical Sciences Research Council) iCASE (Industrial Cooperative Awards in Science \& Technology) studentship, with additional support from Signify Research (née Philips Lighting Research) and The British Museum.

Signify Research provided financial and technical support, with supervision from Kees Teunissen. The British Museum provided technical support and access to facilities, with supervision from Capucine Korenberg. 
\chapter{My First Content Chapter}
\label{chapterlabel2}

% This just dumps some pseudolatin in so you can see some text in place.
\blindtext

\chapter{My Second Content Chapter}
\label{chapterlabel3}

% This just dumps some pseudolatin in so you can see some text in place.
\blindtext

\chapter{General Conclusions}
\label{chap:Conclusions}

\section{Summary of Conclusions}

\begin{enumerate}
\item Assuming the applicability of general damage functions, potential damage may be reduced by using light sources with lower \glspl{CCT} (Figure \ref{fig:CCTvsDI}).
\item Museum professionals currently do not employ this technique, in part due to conflicting results as to how \gls{CCT} interacts with observer preference (Section \ref{sec:CCTmus}).
\item One potential cause for these conflicting results (involvement of melanopsin in colour constancy) was examined with results showing no evidence for a strong or simple interaction (Chapters \ref{chap:LargeSphere} and \ref{chap:SmallSphere}).
\item To address limitations in the performed experiments, and assist with designing future experiments, a computational study was performed to explore whether a melanopsin interaction would be beneficial for colour constancy (Chapter \ref{chap:Melcomp}).
\item In response to this, and observations from psychophysical work, future work is proposed (Section \ref{sec:fut}).
\item A novel method for performing colour constancy experiments with a tablet computer has been provided, and recommendations for further development are suggested (Chapter \ref{chap:Tablet}).
\end{enumerate}

\section{Contributions to Museum Lighting}

The ideal approach would be to find the damage function for the specific material which is of concern, and the \gls{SPD} of the various illuminants under consideration, and compute a range of bespoke damage index values for those specific illuminants and that specific material. This could be achieved using the code provided at \url{https://github.com/da5nsy/DamageIndex}.

However, where this seems like an unreasonable or unpractical undertaking, based on computations performed here (and relying on the applicability of general damage functions) it seems that it would be a wise choice to pick museum lighting of a lower \gls{CCT} when faced with a choice between two otherwise similar light sources.

It is likely that some \glspl{CCT} will be preferred over others. However, it seems likely that luminance and colour rendering are more important factors to visual experience. With this in mind, using the choice of \gls{CCT} as a means to limit damage seems wise.

\section{Contributions to Vision Science}

The psychophysical experiments reported here set out to answer the question `Are \glspl{ipRGC} involved in colour constancy?'. These experiments do not provide evidence of a melanopic input to colour constancy. However, it is noted that these experiments explore only a small fraction of the stimulus space that \glspl{ipRGC} may act over. It is further noted that these experiments, and other experiments done in this field elsewhere, do not necessarily correspond to the conditions under which one may expect to see an effect. The computational study reported in Chapter \ref{chap:Melcomp} aimed to better define what these conditions are most likely to be.

The computational study makes the prediction that colour constancy using a melanopsin-based mechanism should be effective even when there is only one surface present in the scene. Other candidate mechanisms do not share this prediction, and it seems as though this should be a testable hypothesis.

However, since based on the results of \citet{kraft_mechanisms_1999} it seems likely that colour constancy is achieved through the interplay of a great many factors, and there is not a single panacea to the problem of illuminant variance, special care should be taken. It is likely that in some situations certain mechanisms or cues are given weight, where in another situation, others are prioritised.

\section{Future work} \label{sec:fut}

Future work relevant to the computational study and the tablet method is listed at the end of the respective chapters. A number of recommendations for future sphere-type psychophysical investigations are provided below.

\begin{itemize}
\item The experiments reported here relied upon the assumption of cross-retinal adaptation. Whilst this is an interesting area to explore, and certainly worthy of further investigation, a simpler place to start would be to use corresponding locations for adaptation and testing. For example, a screen or other illumination source comprised of a number of concentric circles of different diameters, where the centre and each annulus could be uniquely addressed, would allow an investigator to adapt a particular eccentricity of the retina, and test on that same part of the retina (or a different area). This would have the additional benefit of allowing for observer metameric matches which satisfied the various bands of different spectral sensitivity as eccentricity increases.
\item A valuable concept introduced by \citet{spitschan_selective_2015} is that of `splatter'. This is ``the expected amount of contrast on nominally silenced photoreceptor
classes for a given modulation around a given background'' and is introduced in recognition of the limitations of stimulus control. Including a control condition, or multiple control conditions, where a conservative estimate of the amount of cone/rod contrast which is likely to have avoided silencing is presented, can allow for an effect due to the target variable to be distinguished from noise resulting from the limits of stimulus control. 
\item It is recommended to avoid manual input from observers wherever possible, as this seems to be a considerable source of error. \citet{allen_form_2019} introduced a method for performing perceptual nulling in this type of experiment using an alternative forced choice paradigm, where potential metameric pairs were presented as spatial sinusoidal gratings of one of 4 orientations. An observer had to report the orientation, and the orientations where performance fell to chance levels were selected as `functionally metameric' pairs for each observer. An analogous method to replace the task of achromatic setting might be the method of \citet{smithson_colour_2004}, whereby observers make category judgements rather than tuning the appearance of a stimulus. This has the advantage that stimuli can be shown for a defined amount of time, avoiding the problem whereby the test stimulus affects the observer's state of adaptation. Informal discussion with these investigators has suggested that whilst this method garners data with low levels of noise, it is relatively time-consuming.
\item It does not appear as though this type of experiment is susceptible to the issues identified by \citet{spitschan_selective_2015} whereby cones in the shadow of retinal blood vessels have a different effective spectral sensitivity, and thereby break designed metamerism. This seems only to affect high temporal frequency presentations, which suggests that the perceptual nulling stage may be at risk. \citet{zele_melanopsin_2018} propose a method to avoid this issue by desensitising cones with temporal white noise which is nominally melanopsin silent.
\item The experiments reported here were of relatively low luminance, and personal communications with other researchers in the field have suggested that melanopsin is more likely to be active only at relatively high luminances.
\item In experiments which explore long term adaptation to multiple nominally metameric conditions it would be beneficial to test throughout the experiment whether metamerism was holding. This could be achieved, for example, with brief repeats of the perceptual nulling task, interleaved into the main task.
\item The computational study performed in Chapter \ref{chap:Melcomp} identified that an absolute melanopic signal (`first-level', to use the terminology used in that chapter) may be an inferior target signal compared to a signal which was in some way normalised. The specific normalisation identified in the computational study was a normalisation by luminance, following the template for computing \gls{MB} chromaticity values, though further research is advised before settling on this as a new target variable (it is quite plausible that normalisation to another signal might provide a superior signal).
\item A final note, which is in recognition of a change to standard practice which has occurred during recent years: future experiments of this type (so long as there is a clear model from which a definite prediction can be made) should pre-register the hypothesis and methodology, since it seems likely that this type of experiment (with many uncontrolled / uncontrollable factors and high levels of noise) may be particularly susceptible to post-hoc analysis.
\end{itemize}






\addcontentsline{toc}{chapter}{Appendices}

% The \appendix command resets the chapter counter, and changes the chapter numbering scheme to capital letters.
%\chapter{Appendices}
\appendix

% \chapter{An Appendix About Stuff}
% \label{appendixlabel1}

% \chapter{Another Appendix About Things}
% \label{appendixlabel2}

\chapter{Colophon}
\label{appendixlabel3}

This document has been written with \LaTeX, within a mix of TeXworks (locally) and Overleaf (when online), based on a template created by Ian Kirker\footnote{\url{https://github.com/UCL/ucl-latex-thesis-templates}}.

In addition to the packages used in the template, additional use was made of package `libertine'\footnote{\url{https://ctan.org/pkg/libertine}}, which loads the Linux Libertine and Linux Biolinum font families which are free and open fonts\footnote{\url{http://libertine-fonts.org/}}. %and `microtype' which !!!!!!!!!!!!!!!, also lineno \\
%List other added packages
Zotero and Bib\TeX were used for reference management.

During writing git was used as a version control software, pushing to GitHub. See this thesis as a GitHub repo\footnote{\url{https://github.com/da5nsy/Thesis}}. %at the point of submission and as an updated repo (who knows if I will find typos later!)



%\textit{This is a description of the tools you used to make your thesis. It helps people make future documents, reminds you, and looks good.}
%
%\textit{(example)} This document was set in the Times Roman typeface using \LaTeX\ and Bib\TeX , composed with a text editor. 
 % description of document, e.g. type faces, TeX used, TeXmaker, packages and things used for figures. Like a computational details section.
% e.g. http://tex.stackexchange.com/questions/63468/what-is-best-way-to-mention-that-a-document-has-been-typeset-with-tex#63503

% Side note:
%http://tex.stackexchange.com/questions/1319/showcase-of-beautiful-typography-done-in-tex-friends 
% You could separate these out into different files if you have
%  particularly large appendices.

% This line manually adds the Bibliography to the table of contents.
% The fact that \include is the last thing before this ensures that it
% is on a clear page, and adding it like this means that it doesn't
% get a chapter or appendix number.
\addcontentsline{toc}{chapter}{Bibliography}

% Actually generates your bibliography.
\bibliography{example}

% All done. \o/
\end{document}
