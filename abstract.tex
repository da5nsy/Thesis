\begin{abstract} % 300 word limit

Museums seek a pragmatic compromise between lighting which causes minimal damage to objects and lighting which allows maximal visitor enjoyment. One way to reduce damage is to choose illumination of a lower correlated colour temperature (CCT), since this contains less short-wavelength radiation, which is generally more damaging.

Colour appearance models suggest that a change in CCT alone should not affect visual experience, but there is a long history of studies which aim to find a `preferred' CCT for museum environments. These experiments have often had conflicting findings. One potential cause for this would be if the spectral sensitivity for adaptation differed from the spectral sensitivity for vision, as generally only the chromaticity (not the spectrum) is controlled in such experiments. Such a situation might arise if the recently discovered intrinsically photosensitive Retinal Ganglion Cells were involved in colour constancy.

Two psychophysical experiments were performed whereby observers were adapted to different spectra whilst performing an achromatic matching task. The first experiment used sixteen different narrow-band sources as adapting fields, and the second used two perceptually indistinguishable sources which differed in melanopic power. Neither experiment found evidence for a role of melanopsin in colour constancy, but it was noted that the model of colour constancy implicitly under examination was under-developed and did not produce clear predictions.

Regarding this, an exploratory computational study was performed, to explore whether, and in what way, a melanopsin signal might contribute to an observer's ability to achieve colour constancy. It was found that a normalised melanopic signal provided a means by which colour constancy could be roughly achieved, without the need for scene-level regularities which other algorithms rely upon.

Additionally, a novel method for performing colour constancy experiments upon a tablet computer was developed, so that experiments could be run more easily within real world environments.

\end{abstract}