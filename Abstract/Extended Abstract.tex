\documentclass{article}

\usepackage{libertine}

\title{Museum Lighting, Color Constancy and Melanopsin.}
\date{} 

\begin{document}
\maketitle

\section{Extended Abstract}

Light causes damage to objects in museums. Museums seek a pragmatic compromise between lighting which causes minimal damage to objects, and lighting which allows maximal visitor enjoyment. Generally this is achieved by following industry recommendations for maximum luminance.

A complementary way to reduce damage, highlighted by the Commission Internationale de l'Eclairage in a 2004 publication, is to choose illumination of a lower Correlated Colour Temperature (CCT), because such illumination will contain a higher proportion of longer wavelengths, which are generally less damaging to objects. Computations performed for a range of commercially available lighting products showed a strong correlation between predicted damage and CCT, with a factor of two between the low and high CCT sources. Practically, a sensitive object might be displayed for twice as long under a low CCT illuminant than a high one, before passing a damage threshold. 

A series of interviews with museum professionals showed that this technique is not currently being employed. One reason is a belief that changing the CCT in an environment will affect the visual experience.

Models of vision suggest that a change in CCT alone should not affect visual experience; that we should be able to adapt to any colour of lighting. This seems at most only partially true. It is true that we seem to adapt to the ambient illuminant, both in terms of luminance and chromaticity, such that our perception of object colours relates primarily to object reflectance properties rather than the absolute intensities reaching our eyes. Generally, however, we do also seem to have an awareness of the properties of the illumination in a scene, and even a preference for some illuminants over others. Historically, experiments seeking to find an ideal preferred CCT have provided conflicting results.

One possible reason for these conflicting results might be if a cell group called the `intrinsically photosensitive Retinal Ganglion Cells' (ipRGCs) were involved in colour constancy. This cell group has traditionally been considered as having no output to visual pathways, and so colour constancy experiments haven't controlled for ipRGC activation, but recent research has shown that ipRGCs do in fact have a limited input to visual pathways. There are a range of reasons that suggest that they may play a role in color constancy specifically.

To investigate the effect of different levels of ipRGC activation on an observer's state of colour constancy,	two lab-based psychophysical experiments were performed. Concurrently, a method for performing color constancy experiments outside of the lab environment, which could be completed quickly and with naive observers, was developed. These experiments did not provide evidence for a simple or strong effect of ipRGC activation.

Following these experiments a different approach was adopted: rather than asking what the affect of varying ipRGC activation upon experimental subjects was, instead the question was posed whether an ipRGC-based signal could hypothetically be useful for color constancy, considering what we know of daylight variability and natural surface reflectance properties. A computational methodology was employed.

The applied goal of this research was to increase our understanding of color constancy so as to advise museums on how to reduce damage to objects without degrading visitor experience. Asking whether ipRGCs play a role in color constancy is additionally valuable to the vision science community, and to lighting engineering applications beyond the museum world.


\end{document}