\begin{acknowledgements}

% \begin{itquote}{siegel_another_2012}
% Each day I fail, and each day my experiments are infinitesimally closer to truth.
% \end{itquote}

\begin{citequote}{siegel_another_2012}
The interlocking of friendship and science augurs well, if not just for the science, for the completeness of our lives, our all too limited time as scientists.
\end{citequote}

This PhD program has allowed me many luxuries; an excuse to lounge in the reading room of the Wellcome Collection with an obscure book from one of the many libraries of \acrshort{UCL}, for one, being provided with a light-tight basement (which for some reason no-one else wanted as a work-space) for another.
The greatest luxury however, has been the connections with fellow researchers around the world. At conferences I have found camaraderie and unity in our shared fascination with the field of visual neuroscience. 
\bigskip

\begin{citequote}{minnaert_light_1940}
It is indeed wrong to think that the poetry of Nature's moods in all their infinite variety is lost on one who observers them scientifically, for the habit of observation refines our sense of beauty and adds a brighter hue to the richly coloured background against which each separate fact is outlined.
\end{citequote}

Walking around a sunlit Boston with a friend recently, we agreed that whilst we had both learnt from books, seminars and experiments; a great deal of our learning had come from the years of wandering around the world, experiencing and thinking. Examining our visual systems through our own experiences has helped us immeasurably. This is the mindset afforded by my time at \gls{UCL}, under the encouragement of my supervisors.

I have the unusual privilege to have five supervisors, and each is deserving of my thanks. Stuart Robson (\gls{UCL}), I thank for his calm and level-headed assistance in times of stress, and his confidence in me when it was needed. Kees Teunissen (Signify Research, née Philips Lighting Research) I thank for his patience, on the many occasions where my optimism morphed into tardiness. His scrupulous and careful feedback on my work often asked the hard questions which needed to be asked; generally the very same ones I was trying to avoid. Capucine Korenberg (The British Museum) I thank for her expertise and perseverance, particularly when advising me on the curious world of museum lighting, and arranging for me to be allowed to perform experiments at the British Museum.
Katherine Curran (\gls{UCL}) I thank for her valuable academic contributions as a collaborator, as well as her enduring support, kindness and positivity.
Finally, I thank Lindsay MacDonald, who I met when he was a visiting professor at The University of Westminster (where I completed my undergraduate studies), delivering riveting lectures on colour science. After I finished my undergraduate degree, it was his encouragement that led to me to attend the AIC (Association Internationale de la Couleur) conference in Newcastle, where I confirmed my suspicion that the field of colour research was inhabited by interesting and unusual individuals and that I'd quite like to be part of it. It was at his encouragement again that I applied for the PhD position at \gls{UCL}, and I have benefited from his support throughout, abusing his kindness in lending books from his library, and inheriting his questionably healthy habit of taking on many projects at once. I have had several enjoyable afternoons where we have spent hours in discussion, often drifting from the starting point into the mysterious corners of colour science.

My thanks go to the technical staff at \gls{UCL}, in particularly the Chadwick Workshop technicians, who manufactured all manner of weird and wonderful things for me, the technicians at \gls{UCL} \acrshort{PAMELA} who supported me in using the space, and the staff of the \gls{UCL} Grant Museum of Zoology for allowing me to use the museum for initial experiments. I also wish to note my appreciation and thanks to the various pieces of open source software I used (for further details see Appendix \ref{app:Colophon}).

The research group of 3DImpact has changed greatly since I began. I have made many friends in this group, too many to mention. Thank you for your support and friendship.

My final thanks must go to my partner Alice, without whose patience and support this project would never have reached this point.

\end{acknowledgements}