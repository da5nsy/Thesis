\documentclass{article}

\title{A computational analysis of the value of a melanopic signal to the ecological challenge of colour constancy}
\date{} 

\begin{document}
\maketitle

\section{Justification}

Why are we doing this?
Why do we think it might be valuable?
What is the benefit of knowing whether or not it might be valuable?
Why are we not just testing it psychophysically?

\section{The investigation}
\subsection{The very first question...}
Does the melanopic signal give us a shortcut to the chromaticity of the illuminant? This would be magic, but nature is pretty wonderful.

To answer this question, we loaded the Granada dataset, and calculated chromaticity values and melanopic values for each illuminant. 

plot: correlation between MB1 and I, and MB2 and I.

There is something of a regularity, but it doesn't initially appear to be a useful relationship. But this isn't really surprising because chromatic signals are inherently different to simple linear computed cone signals. One samples a specific point of the spectrum, and the other is a comparison between two different parts of the spectrum.

Therefore, even the signals from which the chromatic signals are derived, bear little correlation to the chromatic signals which they produce.

plot: other cone signals.

Thought of another way: you could say that the simple linear signals are mostly correlated with the overall amount of illumination, and the comparative signals are more correlated with the skewdness of the illumination spectra. --> principal components.

So let's ask another question: what is the optimal way to bake a signal which predicts the second principal component of daylight? (This is of particular interest because not only does predicting the pc scores of an illum efficiently tell us information about the illumination, but it can also theoretically be used to predict the corresponding pc scores for reflectances in a scene).

Well, presumably you'd want one which compared two signals at extreme points of the pc2 spectrum. (Running a simulation of purely hypothetical sensors confirms this?)

plot: do this? 

But this thinking is slightly flawed. So far we haven't considered reflectances.


(note: 'the chromaticity of the illuminant' and 'the chromatic shift required to correct the appearance of objects under an illuminant' are subtly different concepts. See Maloney and the RGB phalacy. Or some title to that effect)

If we consider reflectances then we get a perturbed version of the above.

\section{Exploratory research}
\section{Reproducible research statement}




\end{document}
