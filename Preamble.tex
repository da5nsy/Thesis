% I may change the way this is done in a future version, 
%  but given that some people needed it, if you need a different degree title 
%  (e.g. Master of Science, Master in Science, Master of Arts, etc)
%  uncomment the following 3 lines and set as appropriate (this *has* to be before \maketitle)
% \makeatletter
% \renewcommand {\@degree@string} {Master of Things}
% \makeatother

\title{Museum Lighting, Colour Constancy and Melanopsin}
\author{Daniel Garside}
\department{Civil, Environmental and Geomatic Engineering}

\maketitle
\makedeclaration

\begin{abstract} % 300 word limit

Insert abstract

% Extended abstract: 

%Light causes damage to objects in museums. Museums seek a pragmatic compromise between lighting which causes minimal damage to objects, and lighting which allows maximal visitor enjoyment. Generally this is achieved by following industry recommendations for maximum illuminance and colour rendering index.
%
%A complementary way to reduce damage, highlighted by the Commission Internationale de l'Eclairage in a 2004 publication, is to choose illumination of a lower Correlated Colour Temperature (CCT), because such illumination will contain most radiation energy in the longer wavelengths, which are generally less damaging to objects. Computations performed for a range of commercially available lighting products showed a strong correlation between predicted damage and CCT, with a factor of two between the low and high CCT sources. Practically, a sensitive object might be displayed for twice as long under a low CCT illuminant than a high one, before passing a damage threshold. 
%
%A series of interviews with museum professionals showed that this technique is not currently being employed. One reason is a belief that changing the CCT in an environment will affect the visual experience and the atmosphere in the room.
%
%Models of vision suggest that a change in CCT alone should not affect visual experience; that in a space with a single type of lighting we should be able to adapt to any colour of illumination. This seems at most only partially true. It is true that we adapt reasonably well to the ambient illumination, both in terms of luminance and chromaticity, such that our perception of object colours relates primarily to object reflectance properties rather than the absolute intensities reaching our eyes. Generally, however, we do also seem to have an awareness of the properties of the illumination in a scene, and even a preference for some sources over others. Historically, experiments seeking to find an ideal preferred CCT have provided conflicting results.
%
%One possible reason for these conflicting results might be that an additional retinal mechanism is involved in chromatic adaptation. The studies in this thesis have investigated whether a cell group called the `intrinsically photosensitive Retinal Ganglion Cells' (ipRGCs) might be involved in colour constancy. This cell group has traditionally been considered as having no output to visual pathways, and so colour constancy experiments haven't controlled for ipRGC activation, but recent research has shown that ipRGCs do in fact have a limited input to visual pathways. There are a range of reasons that suggest that they may play a role in colour constancy specifically.
%
%To investigate the effect of different levels of ipRGC activation on an observer's state of colour constancy,	two lab-based psychophysical experiments were performed. The first sought to examine the effect of different wavelengths of light upon chromatic adaptation. Within a Ganzfeld viewing environment, illuminated by one of 16 different wavelengths of near-monochromatic light, observers performed an achromatic setting task, controlling the chromaticity of a display visible in the central field through a 4° circular aperture with two handheld sliders.
%
%In the second experiment the role of melanopsin in chromatic adaptation was more directly questioned. The same task was performed as in the first experiment, but the Ganzfeld was this time illuminated by one of two perceptually metameric lights with different melanopic irradiance levels. Neither of these experiments provided evidence for a simple or strong effect of ipRGC activation, though this is limited to the case where peripheral stimulation would affect central perception.
%
%Concurrently, a method has been developed for performing colour constancy experiments outside of the lab environment, which can be completed quickly by naive observers in `real-life' illumination conditions. The method uses a tablet computer, on which an isoluminant plane through CIELUV is presented, successively varying in orientation and spatial offset. The observer is tasked with selecting, by touching with a finger, an achromatic point from within each stimulus. From the recorded selections an estimate of the observer's state of chromatic adaptation is computed.
%
%Following these experiments a different approach was adopted: rather than asking what was the affect of varying ipRGC activation upon experimental subjects, instead the question was posed whether an ipRGC-based signal could hypothetically be useful for colour constancy, considering what we know of daylight variability and natural surface reflectance properties. A computational methodology was employed. It was found that the spectral sensitivity of melanopsin is optimal for providing a signal that can transform raw chromatic signals to an illuminant-independent space, and that this can be done without using any scene level assumptions such as grey-world.
%
%The applied goal of this research was to increase our understanding of colour constancy so as to advise museums on how to reduce damage to objects without degrading visitor experience. Asking whether ipRGCs play a role in colour constancy is additionally valuable to the vision science community, and to lighting engineering applications beyond the museum world.
\end{abstract}

\begin{impactstatement}

Insert Impact Statement

%	UCL theses now have to include an impact statement. \textit{(I think for REF reasons?)} The following text is the description from the guide linked from the formatting and submission website of what that involves. (Link to the guide: {\scriptsize \url{http://www.grad.ucl.ac.uk/essinfo/docs/Impact-Statement-Guidance-Notes-for-Research-Students-and-Supervisors.pdf}})
%
%\begin{quote}
%The statement should describe, in no more than 500 words, how the expertise, knowledge, analysis,
%discovery or insight presented in your thesis could be put to a beneficial use. Consider benefits both
%inside and outside academia and the ways in which these benefits could be brought about.
%
%The benefits inside academia could be to the discipline and future scholarship, research methods or
%methodology, the curriculum; they might be within your research area and potentially within other
%research areas.
%
%The benefits outside academia could occur to commercial activity, social enterprise, professional
%practice, clinical use, public health, public policy design, public service delivery, laws, public
%discourse, culture, the quality of the environment or quality of life.
%
%The impact could occur locally, regionally, nationally or internationally, to individuals, communities or
%organisations and could be immediate or occur incrementally, in the context of a broader field of
%research, over many years, decades or longer.
%
%Impact could be brought about through disseminating outputs (either in scholarly journals or
%elsewhere such as specialist or mainstream media), education, public engagement, translational
%research, commercial and social enterprise activity, engaging with public policy makers and public
%service delivery practitioners, influencing ministers, collaborating with academics and non-academics
%etc.
%
%Further information including a searchable list of hundreds of examples of UCL impact outside of
%academia please see \url{https://www.ucl.ac.uk/impact/}. For thousands more examples, please see
%\url{http://results.ref.ac.uk/Results/SelectUoa}.
%\end{quote}
\end{impactstatement}

\begin{acknowledgements}
Acknowledge all the things!

I also wish to note my appreciation and thanks to the various pieces of open source software I used in the writing of this thesis. For further details see \ref{appendixlabel3}.
\end{acknowledgements}

\setcounter{tocdepth}{2} 
% Setting this higher means you get contents entries for
%  more minor section headers.

\tableofcontents
\listoffigures
%\listoftables

