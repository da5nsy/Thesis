\subsection{Intrinsically Photosensitive Retinal Ganglion Cells (ipRGCs)}

\textit{A recent review by \citet{spitschan_melanopsin_2019} provides an overview of this rapidly progressing research area.}

\bigskip

Retinal rod and cone cells (of three types, l/m/s) are well established as the primary receptors for human vision, and their connections and properties are relatively well understood. Two distinct signalling pathways originate from the retina, one which is associated with image formation and the other which is considered to be \gls{NIF}, and which influences systems such as circadian rhythm entrainment, pupillary reflex and melatonin release. It was originally thought that rods and cones were the sole inputs to both of these pathways\citep{hankins_melanopsin_2008}.

\Glspl{RGC} combine signals from groups of cones and rods and relay these signals via the optic nerve to the lateral geniculate nucleus, which in turn processes and relays them further to the cortex for additional processing, allowing for classical vision of objects, movement and colour. \Glspl{ipRGC} are a sub-class of retinal ganglion cells, which in addition to combining and relaying signals exhibit some intrinsic photosensitivity of their own. 

This intrinsic photosenstivity was only confirmed recently\citep{qiu_induction_2005} relative to our knowledge of other retinal cell types, following a search for a retinal cell type or combination of cell types which would fit the spectral sensitivity properties found to influence entraining the circadian rhythm in humans and other animals\citep{brainard_human_2001,brainard_action_2001}, which was dissimilar to all of the spectral sensitivities of the cell classes known at the time.

Additionally, it was found that animals and humans with no functioning cones or rods were still able to have a correctly functioning circadian system\citep{freedman_regulation_1999,zaidi_short-wavelength_2007}, further suggesting that that the circadian rhythm was influenced by a novel receptor with a distinct photoreceptor. It is now believed to be these cells which provide input to the second of the signalling pathways originating in the retina, that which controls the \gls{NIF} response to illumination.

\Glspl{ipRGC} were found to express a photopigment fitting such attributes, named melanopsin. The spectral sensitivity of melanopsin peaks around 480nm\citep{qiu_induction_2005,hankins_primary_2002,dacey_melanopsin-expressing_2005,peirson_melanopsin_2006,bailes_human_2013} which places it between the s-cone (cyanolabe photopsin) and rod cell (rhodopic rhodopsin) peak spectral sensitivities, see Figure \ref{fig:specsens}. In humans, pre-receptoral filtering leads to a functional peak sensitivity of closer to 490nm\citep{cie_cie_2015-1}. There is uncertainty regarding the potential bistable nature of melanopsin, and whether this would have a functional effect on the spectral sensitivity of melanopsin\citep{cie_cie_2015-1,mure_melanopsin_2009,rollag_does_2008}.

\Glspl{ipRGC} are sparse in the retina; discounting input from other cell types, they operate at a much lower resolution than as would be required for meaningful spatial vision. They also operate much more slowly compared to other cell types, taking several seconds to respond, but are able to sustain a response in contrast to other retinal cell types which are able to respond quickly but only for short periods.

It has been proposed that \glspl{ipRGC} may allow an observer to sense a level of absolute irradiance\citep{brown_melanopsin_2010}. 

\subsubsection{The image-forming role of ipRGCs}

In recent years there has been a large number of publications examining the role of melanopsin outside of \gls{NIF} vision.

Spatial:
\cite{ecker_melanopsin-expressing_2010}
\cite{spitschan_vision_2017}
\cite{mouland_responses_2017}
\cite{allen_melanopsin_2017}
\cite{allen_form_2019}

Colour:
\cite{cao_evidence_2018}
\cite{spitschan_human_2017-1}
\cite{zele_melanopsin_2018}