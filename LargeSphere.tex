\chapter{Large Sphere}
\label{chap:LargeSphere}

\textit{The work presented here has been presented previously as an oral presentation at AIC 2013 \citep[p. 623]{macdonald_chromatic_2013} \textbf{prior to the author's involvement}, and as an oral presentation at ICVS 2017 \citep[p. 35/58]{jan_kremers_24th_2017} by the author.}

\section{Summary}

The goal of this experimental work was to examine the effect of different wavelengths of light upon chromatic adaptation. Our hypothesis is that \gls{ipRGC} stimulation may need to be considered in order to fully model the induced adaptation, with the null hypothesis being that chromatic adaptation can be fully accounted for cone and rod mechanisms.

Within a Ganzfeld viewing environment, illuminated by one of 16 different wavelengths of near-monochromatic light, observers performed an achromatic setting task, controlling the chromaticity of a display visible in the central field through a 4$^{\circ}$ circular aperture with two handheld sliders.

Results:

This project was designed before the author arrived at \gls{UCL}, and data from two participants had already been collected. Data collection required at least 16 hours commitment from observers, and so the only observers up to that point had been LM (one of the authors academic supervisors), who initiated the experiment, and TR who \dots %%%
The original goal for my involvement in this project was that I should be a third observer, and assist in the data analysis. However, following the collection and initial data analysis of my own data, it became clear that there had been a technical fault during this latest run of data collection, and my data was excluded. This data is discussed further in Appendix X. %%%%%%%%
Thus, my only contribution to this work is an extension to the data analysis started by LM, which I shall focus on in this chapter.

\section{Methodology}


Hardware
Observations
Data Processing

\section{Results}
\section{Discussion}
\section{Conclusion}