\chapter{Introduction}
\label{chapterlabel1}

\section{Context}
Light causes damage to objects in museums. Museums seek a pragmatic compromise between lighting which causes minimal damage to objects, and lighting which allows maximal visitor enjoyment. Generally this is achieved by following industry recommendations for maximum illuminance and \gls{CRI}.

A complementary way to reduce damage, highlighted by the \gls{CIE} in a 2004 publication \citep{cie_cie_2004}, is to choose illumination of a lower \gls{CCT}, because such illumination will contain most radiation energy in the longer wavelengths, which are generally less damaging to objects. To verify and extend the work of the \gls{CIE}, a \gls{DI} was calculated for a range of commercially available lighting products (\textbf{Section \ref{sec:CCTmus}}). This showed a strong correlation between predicted damage and \gls{CCT}, with a factor of two between the low and high \gls{CCT} sources. Practically, a sensitive object might be displayed for twice as long under a low \gls{CCT} illuminant than a high one, before passing a particular damage threshold. 

A series of interviews with museum professionals showed that this technique is not currently being employed (\textbf{Chapter \ref{chap:Interviews}}). One reason is a belief that changing the \gls{CCT} in an environment will affect the visual experience and the atmosphere in the room.

Most models of vision suggest that a change in \gls{CCT} alone, within moderate limits, should not affect visual experience; that in a space with a single type of lighting we should be able to adapt to any colour of illumination. This seems at most only partially true. It is true that we adapt reasonably well to the ambient illumination, both in terms of luminance and chromaticity, such that our perception of object colours relates primarily to object reflectance properties rather than the absolute intensities reaching our eyes. Generally, however, we do also seem to have an awareness of the properties of the illumination in a scene, and even a preference for some sources over others. Historically, experiments seeking to find an ideal preferred \gls{CCT} have provided conflicting results.

One possible reason for these conflicting results might be that an additional retinal mechanism is involved in chromatic adaptation. The studies in this thesis have investigated whether a cell group called the \glspl{ipRGC} might be involved in colour constancy. This cell group has traditionally been considered as having no output to visual pathways, and so colour constancy experiments haven't controlled for \gls{ipRGC} activation, but recent research has shown that \glspl{ipRGC} do in fact have a limited input to visual pathways. There are a range of reasons that suggest that they may play a role in colour constancy specifically.

\section{Research Question}

The principal research question considered in this thesis is: \textbf{Are \glspl{ipRGC} involved in colour constancy?}

If they are, this may explain why previous investigations into preferred \gls{CCT} have produced conflicting results. If we can understand how they are involved, this may give us a new insight into how to vary the \gls{CCT} of museum illumination in such a way that we can limit damage without degrading visitor experience.

\section{Chapter Summaries}

To better understand how museum professionals currently make lighting decisions (what metrics they use, what the guidance is for using these metrics, what the current challenges and limitations are) a series of interviews was performed with museum lighting professionals \citep{garside_how_2017}. The responses to these interviews are summarised in \textbf{Chapter \ref{chap:Interviews}}.

To investigate the effect of different levels of \gls{ipRGC} activation on an observer's state of colour constancy, two lab-based psychophysical experiments were performed (\textbf{Chapters \ref{chap:LargeSphere} and \ref{chap:SmallSphere}}). The first sought to examine the effect of different wavelengths of light upon chromatic adaptation. Within a Ganzfeld viewing environment, illuminated by one of 16 different wavelengths of near-monochromatic light, observers performed an achromatic setting task, controlling the chromaticity of a display visible in the central field through a 4$^{\circ}$ circular aperture with two handheld sliders.

In the second experiment the role of melanopsin in chromatic adaptation was more directly questioned. The same task was performed as in the first experiment, but the Ganzfeld was this time illuminated by one of two perceptually metameric lights with different melanopic illuminance levels. Neither of these experiments provided evidence for a simple or strong effect of ipRGC activation.

Concurrently, a method has been developed for performing colour constancy experiments outside of the lab environment, which can be completed quickly by naive observers in `real-life' conditions (\textbf{Chapter \ref{chap:Tablet}}). The method uses a tablet computer, on which an isoluminant plane through CIELUV is presented, successively varying in orientation and spatial offset. The observer is tasked with selecting, by touching with a finger, an achromatic point from within each stimulus. From the recorded selections an estimate of the observer's state of chromatic adaptation is computed.

Following these experiments a different approach was adopted: rather than asking what was the affect of varying \gls{ipRGC} activation upon experimental subjects, instead the question was posed whether an \gls{ipRGC}-based signal could hypothetically be useful for colour constancy, considering what we know of daylight variability and natural surface reflectance properties (\textbf{Chapter \ref{chap:Melcomp}}). A computational methodology was employed for this exploratory study. It was found that a simple transform can be made from raw signals to an illuminant independent space, with the use of a melanopsin-based signal. Further, it was found that the spectral sensitivity of melanopsin is near-optimal for providing a signal for such a transformation, and that this can be done without using any scene level assumptions such as `grey-world'.

Chapter \ref{chap:Conclusions} provides a summary and discussion of results, and suggests routes for future work.

\section{Associated Publications}

{\parindent0pt \footnotesize


\subsection*{Journal Papers}

\textbf{Daniel Garside}, Katherine Curran, Capucine Korenberg, Lindsay MacDonald, Kees Teunissen, and Stuart Robson. How is museum lighting selected? An insight into current practice in UK museums. \textit{Journal of the Institute of Conservation}, 40(1):3-14, January 2017. \doi{10.1080/19455224.2016.1267025}

\subsection*{Conference Papers}

\textbf{Daniel Garside}. Initial Observations on Lighting Situations in The British Museum. In \textit{1st International Conference, Science and Engineering in Arts, Heritage and Archaeology}, page 58, London, UK, July 2015. \doi{10.6084/m9.figshare.4269665.v1}
\bigskip

\textbf{Daniel Garside}, Katherine Curran, Capucine Korenberg, Lindsay MacDonald, Kees Teunissen, and Stuart Robson. Interviewing Museum Professionals: How is Museum Lighting Selected? In \textit{SEAHA (Science and Engineering in Arts, Heritage and Archaeology)}, Oxford, UK, June 2016. \doi{10.6084/m9.figshare.5840952.v1}
\bigskip

\textbf{Daniel Garside}, Lindsay MacDonald, Kees Teunissen, Katherine Curran, Capucine Korenberg, and Stuart Robson. Potential uses for spectrally variable lighting in museum environments. In \textit{Progress in Colour Studies (PICS)}, page 100, London, UK, September 2016. \doi{10.6084/m9.figshare.4269659.v1}
\smallskip %This one appears to create a bigger gap for some reason

\textbf{Daniel Garside}, Lindsay MacDonald, Kees Teunissen, and Stuart Robson. Estimating Chromatic Adaptation in a Museum Environment Using a Tablet Computer. In
\textit{Proceedings of AIC 2016 Interim Meeting - Color in Urban Life: Images, Objects and Spaces}, pages 125-129, Santiago, Chile, October 2016. \doi{10.6084/m9.figshare.4269680.v2}
\bigskip

Lindsay MacDonald and \textbf{Daniel Garside}. Adapting to a Chromatic Environment In \textit{Proceedings of AIC 2016 Interim Meeting - Color in Urban Life: Images, Objects and Spaces}, pages 109-113, Santiago, Chile, October 2016.
\bigskip

\textbf{Daniel Garside} and Lindsay MacDonald. Investigations into the effect of different spectra upon the process of chromatic adaption. In \textit{Proceedings of the International Colour Vision Symposium (ICVS)}, page 35, Erlangen, Germany, August 2017.
\bigskip

\textbf{Daniel Garside}. Light art as a pedagogical tool for teaching the science of colour perception. In \textit{Abstracts from the 5th Visual Science of Art Conference (VSAC)}, page 381, Berlin, Germany, August 2017. \doi{10.1163/22134913-00002099}
\bigskip

\textbf{Daniel Garside}, Stuart Robson, Lindsay MacDonald, Katherine Curran, Kees Teunissen, and Capucine Korenberg.  A method for performing colour constancy studies using a tablet computer. In \textit{European Conference on Visual Perception (ECVP)}, Berlin, Germany, August 2017. \doi{10.6084/m9.figshare.5478493.v1}
\bigskip

M. Hess, \textbf{D. Garside}, T. Nelson, S. Robson, and T. Weyrich.  Object-Based Teaching and Learning for a Critical Assessment of Digital Technologies in Arts and Cultural Heritage. In \textit{ISPRS - International Archives of the Photogrammetry, Remote Sensing and Spatial Information Sciences, volume XLII-2-W5}, pages 349-354, Ottawa, Canada, August 2017. Copernicus GmbH. \doi{10.5194/isprs-archives-XLII-2-W5-349-2017}
\bigskip

L. W. MacDonald, \textbf{D. Garside}, and C. Teunissen. Melanopsin and Colour Vision. In \textit{Proceedings of 13th AIC Congress}, Jeju, South Korea, Oct 2017.
\bigskip

E. K. Webb, S. Robson, L. MacDonald, \textbf{D. Garside}, and R. Evans. Spectral and 3d Cultural Heritage Documentation Using a Modified Camera. In \textit{ISPRS - International Archives of the Photogrammetry, Remote Sensing and Spatial Information Sciences, volume XLII-2}, pages 1183-1190, Riva del Garda, Italy, May 2018. \doi{10.5194/isprs-archives-XLII-2-1183-2018}
\bigskip

\textbf{Daniel Garside}. Is there a role for melanopsin in chromatic adaptation? At \textit{The 16th International Symposium on the Science and Technology of Lighting (LS16)}, Sheffield, UK, June 2018.
\bigskip


\textbf{Daniel Garside}, Lindsay MacDonald, and Kees Teunissen. Does the spectral sensitivity of melanopsin in ipRGCs suggest a role in chromatic adaptation?  In \textit{Vision Sciences Society Annual Meeting Abstract, volume 18}, page 877, St Pete Beach, Florida, September 2018 \doi{10.1167/18.10.877}
\bigskip

\textbf{Daniel Garside} and Lindsay MacDonald. Assessing the effectiveness of a melanopsin-based signal for colour constancy. In \textit{Proceedings of the International Colour Vision Symposium (ICVS)}, page 65, Riga, Latvia, July 2019. 
\bigskip

\subsection*{Editorial Contributions}
A. Pokorska, P. Andrikopoulos, \textbf{D. Garside}, and C. Coon, editors. Book of Abstracts, Museum Lighting Symposium and Workshops.  \textit{1st International Museum Lighting Symposium \& Workshops}, London, UK, September 2017. \doi{10.14324/000.bk.10048078}

}

\section{Affiliations}
This project was supported by an EPSRC (Engineering and Physical Sciences Research Council) iCASE (Industrial Cooperative Awards in Science \& Technology) studentship, with industry sponsors Signify Research (née Philips Lighting Research) and The British Museum.

Signify Research provided financial and technical support, with supervision from Kees Teunissen. The British Museum provided technical support and access to facilities, with supervision from Capucine Korenberg. 